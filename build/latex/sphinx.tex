%% Generated by Sphinx.
\def\sphinxdocclass{report}
\documentclass[a4paper,12pt,english]{sphinxmanual}
\ifdefined\pdfpxdimen
   \let\sphinxpxdimen\pdfpxdimen\else\newdimen\sphinxpxdimen
\fi \sphinxpxdimen=.75bp\relax
%% turn off hyperref patch of \index as sphinx.xdy xindy module takes care of
%% suitable \hyperpage mark-up, working around hyperref-xindy incompatibility
\PassOptionsToPackage{hyperindex=false}{hyperref}

\PassOptionsToPackage{warn}{textcomp}

\catcode`^^^^00a0\active\protected\def^^^^00a0{\leavevmode\nobreak\ }
\usepackage{cmap}
\usepackage{fontspec}
\defaultfontfeatures[\rmfamily,\sffamily,\ttfamily]{}
\usepackage{amsmath,amssymb,amstext}
\usepackage{polyglossia}
\setmainlanguage{english}



\setmainfont{TeX Gyre Termes}
\setsansfont{TeX Gyre Heros}
\setmonofont{TeX Gyre Cursor}
    

\usepackage[Bjornstrup]{fncychap}
\usepackage{sphinx}

\fvset{fontsize=\small}
\usepackage{geometry}


% Include hyperref last.
\usepackage{hyperref}
% Fix anchor placement for figures with captions.
\usepackage{hypcap}% it must be loaded after hyperref.
% Set up styles of URL: it should be placed after hyperref.
\urlstyle{same}

\addto\captionsenglish{\renewcommand{\contentsname}{สารบัญ}}

\usepackage{sphinxmessages}
\setcounter{tocdepth}{0}


\XeTeXlinebreaklocale ”th”
\XeTeXlinebreakskip = 0pt plus 0pt

\usepackage{fontspec}
\defaultfontfeatures{Mapping=tex-text}

\newfontfamily{\thaifont}[Scale=MatchUppercase,Mapping=textext]{TH Sarabun New}
\newenvironment{thailang}{\thaifont}{}
\usepackage[Latin,Thai]{ucharclasses}

\setTransitionTo{Thai}{\begin{thailang}}
\setTransitionFrom{Thai}{\end{thailang}}

\usepackage{setspace}

\usepackage{polyglossia}
\setdefaultlanguage{english}
\setotherlanguage{thai}

\AtBeginDocument\captionsthai % Force the caption to Thai
    

\title{ภาระงานบุคลากรสายวิชาการ}
\date{ก.พ. 24, 2021}
\release{}
\author{คณะวิทยาศาสตร์และเทคโนโลยี มหาวิทยาลัยราชมงคลพระนคร}
\newcommand{\sphinxlogo}{\vbox{}}
\renewcommand{\releasename}{}
\makeindex
\begin{document}

\pagestyle{empty}
\sphinxmaketitle
\pagestyle{plain}
\sphinxtableofcontents
\pagestyle{normal}
\phantomsection\label{\detokenize{index::doc}}


คณะวิทยาศาสตร์และเทคโนโลยี มหาวิทยาลัยเทคโนโลยีราชมงคลพระนคร


\chapter{การนำส่งหลักฐานการประเมินผลสัมฤทธิ์ของงานของพนักงานมหาวิทยาลัย (องค์ประกอบที่ ๑)  ตำแหน่งวิชาการ}
\label{\detokenize{submission_part1:id1}}\label{\detokenize{submission_part1::doc}}
คะแนนการประเมิน แบ่งได้ดังต่อไปนี้
\begin{enumerate}
\sphinxsetlistlabels{\arabic}{enumi}{enumii}{}{.}%
\item {} 
งานสอน ร้อยละ 50

\item {} 
งานวิจัยและงานวิชาการอื่น ร้อยละ 20

\item {} 
งานบริการวิชาการ ร้อยละ 15

\item {} 
งานทำนุบำรุงศิลปวัฒนธรรม ร้อยละ 5

\item {} 
งานอื่น ๆ ร้อยละ 10

\end{enumerate}

ผลงานของผู้ดำรงตำแหน่งทางวิชาการ ตาม ประกาศมหาวิทยาลัยเทคโนโลยี
ราชมงคลพระนคร เรื่อง เกณฑ์ภาระงานขั้นต่ำฯ พ.ศ. 2559 (ข้อ 5)

ใช้หลักฐานที่เป็นรูปแบบไฟล์อิเล็กทรอนิกส์ (PDF หรือ ภาพถ่าย) ตามหัวข้อผลงานในประกาศมหาวิทยาลัยเทคโนโลยีราชมงคลพระนคร เรื่อง เกณฑ์ภาระงานขั้นต่ำของคณาจารย์ประตำ พ.ศ. 2559 ข้อ 5 1. ผลงานของผู้ดำรงตำแหน่งทางวิชาการ
\begin{itemize}
\item {} 
ในส่วนของบทความจากผลงานวิจัย และ/หรือ บทความทางวิชาการ ซึ่งการทำข้อตกลงคณาจารย์ต้องระบุ 100\% เท่านั้น

\item {} 
ส่วนผลงานอื่นสามารถระบุ \% ในการจัดทำผลงานทำต่อปี

\item {} 
ผลงานของผู้ดำรงตำแหน่งทางวิชาการที่ระบุไม่ถึง 100\% ไม่ต้องดำเนินการขอเผยแพร่ แต่ให้จัดทำบันทึกข้อความเสนอคณบดีรับทราบ

\end{itemize}

\begin{sphinxadmonition}{warning}{Warning:}
การส่ง Draft Manuscript ไม่สามารถเคลมภาระงานได้
\end{sphinxadmonition}


\bigskip\hrule\bigskip



\section{1. งานสอน}
\label{\detokenize{submission_part1:id2}}

\subsection{ระดับความสำเร็จในการจัดทำ}
\label{\detokenize{submission_part1:id3}}\begin{description}
\item[{ระดับที่ ๑}] \leavevmode
มี มคอ.๓ และ/หรือ มคอ.๔ ประจำรายวิชาสอนที่เป็นไปตามข้อกำหนด/ตามแบบฟอร์มที่ มทร.พระนคร กำหนด หรือมีโครงการสอนในหลักสูตรที่ไม่ใช่หลักสูตร TQF

\item[{ระดับที่ ๒}] \leavevmode
เป็นไปตามระดับที่ ๑ และมีการพัฒนาสื่อการสอนประกอบโครงการสอน หรือ มีการเรียนการสอนตาม มคอ. ๓ และ/หรือ มคอ.๔

\item[{ระดับที่ ๓}] \leavevmode
เป็นไปตามระดับที่ ๒ และมีการสอบวัดผลการศึกษาตามระเบียบของ มหาวิทยาลัย และประกาศมหาวิทยาลัยเทคโนโลยีราชมงคลพระนคร เรื่องเกณฑ์การวัดและประเมินผล

\item[{ระดับที่ ๔}] \leavevmode
เป็นไปตามระดับที่ ๓  และมีการจัดการเรียนการสอนที่เน้นผู้เรียนเป็นสำคัญ อาทิเช่น การสอนแบบแก้ปัญหา รูปแบบการเรียนที่ใช้
ปัญหาเป็นหลัก วิธีสอนแบบระดมพลังสมอง วิธีสอนแบบบูรณาการ ฯลฯ

\item[{ระดับที่ ๕}] \leavevmode
เป็นไปตามระดับที่ ๔ และมีการจัดทำ มคอ.๕ และ/หรือ มคอ.๖ รวมทั้งมีการนำผลไปปรับปรุงการสอน

\end{description}


\subsubsection{มคอ. 3/4/5/6}
\label{\detokenize{submission_part1:id4}}
ใช้หลักฐานจากระบบบริการการศึกษา เมนูภาระงาน มคอ. เมนูย่อยบันทึก มคอ.3 /มคอ.4  และ มคอ.5/มคอ.6 ร่วมกับ ข้อมูลของฝ่ายวิชาการ วิจัยและบริการวิชาการ

หากมีการรายงานคณาจารย์ดำเนินการส่ง มอค. 3, 4, 5, 6 ช้ากว่าที่มหาวิทยาลัยเทคโนโลยี
ราชมงคลพระนครกำหนดไว้ให้หัวหน้าสาขาวิชา/หัวหน้าหมวดวิชาดำเนินการพิจารณาประเมินพฤติกรรมการปฏิบัติราชการ (องค์ประกอบที่ 2)


\subsubsection{สื่อการสอน}
\label{\detokenize{submission_part1:id5}}\begin{itemize}
\item {} 
ส่งหลักฐานสื่อการสอน \sphinxstylestrong{ทุก} วิชา ตามตารางสอนในรอบการประเมิน

\item {} 
ระบุชื่อของอาจารย์ผู้สอน และสาขาวิชาหรือหมวดวิชาในสื่อการสอนให้ชัดเจน

\item {} 
รายละเอียดหลักฐานของสื่อการสอนแต่ละประเภท มีดังนี้

\end{itemize}
\begin{enumerate}
\sphinxsetlistlabels{\arabic}{enumi}{enumii}{}{.}%
\item {} 
เอกสารประกอบการสอน เอกสารคำสอน หนังสือ และ ตำรา

\end{enumerate}
\begin{itemize}
\item {} 
แบบรับรองการเผยแพร่ผลงานทางวิชาการ  หรือ

\item {} 
แบบรับรองภาระงานทางวิชาการ  หรือ

\item {} 
หน้าปก คำนำ และสารบัญ

\end{itemize}
\begin{enumerate}
\sphinxsetlistlabels{\arabic}{enumi}{enumii}{}{.}%
\setcounter{enumi}{1}
\item {} 
เอกสารนำเสนอ(พาวเวอร์พอยท์)

\end{enumerate}
\begin{itemize}
\item {} 
หน้าปก และ เนื้อหาจำนวน ไม่ต่ำกว่า 1 หน่วยเรียน

\end{itemize}
\begin{enumerate}
\sphinxsetlistlabels{\arabic}{enumi}{enumii}{}{.}%
\setcounter{enumi}{2}
\item {} 
บทเรียนในระบบอินเตอร์เน็ต
\begin{itemize}
\item {} 
ลิงก์ที่สามารเข้าถึงไฟล์อิเล็กทรอนิกส์ของเนื้อหาที่ระบุในบทเรียนในระบบอินเตอร์เน็ต

\end{itemize}

\item {} 
หนังสือ กรณีได้รับการจัดพิมพ์และจัดจำหน่ายโดยสำนักพิมพ์ที่ได้รับรองทางวิชาการ
\begin{itemize}
\item {} 
หลักฐานว่าได้ผ่านการประเมินโดยคณะผู้ทรงคุณวุฒิในสาขาวิชานั้น ๆ หรือสาขาวิชาที่เกี่ยวข้อง (peer reviewer) ที่มาจากหลากหลายสถาบัน และ

\item {} 
หน้าปกนอก และ หน้าแสดงลิขสิทธิ์และบรรณานุกรมของหนังสือ (CIP)

\end{itemize}

\end{enumerate}


\subsubsection{นิยามของสื่อการสอน}
\label{\detokenize{submission_part1:id6}}

\bigskip\hrule\bigskip



\section{2. งานวิจัยและงานวิชาการอื่น}
\label{\detokenize{submission_part1:id7}}
เลือกข้อใดข้อหนึ่งระหว่าง
* 2.1 งานวิจัย
* 2.2 งานวิชาการอื่น


\subsection{ระดับความสำเร็จในการจัดทำงานวิจัย}
\label{\detokenize{submission_part1:id8}}\begin{description}
\item[{ระดับที่ ๑}] \leavevmode
มีหัวข้องานวิจัยหรือข้อมูลเบื้องต้นที่จะนำไปสู่หัวข้องานวิจัย

\item[{ระดับที่ ๒}] \leavevmode
เป็นไปตามระดับที่ ๑ และเสนอโครงการวิจัยต่อหน่วยงาน

\item[{ระดับที่ ๓}] \leavevmode
เป็นไปตามระดับที่ ๒ และได้รับการอนุมัติให้ดำเนินการวิจัย

\item[{ระดับที่ ๔}] \leavevmode
เป็นไปตามระดับที่ ๓ และดำเนินการวิจัยเสร็จตามกำหนด

\item[{ระดับที่ ๕}] \leavevmode
เป็นไปตามระดับที่ ๔ และมีการเผยแพร่ผลงานวิจัย/การจัดนิทรรศการ/การนำเสนอผลงาน ในที่ประชุมระดับชาติ หรือนานาชาติ

\end{description}


\subsection{ระดับความสำเร็จในการจัดทำงานวิชาการอื่น}
\label{\detokenize{submission_part1:id9}}
\begin{sphinxadmonition}{note}{Note:}
ผลงานทางวิชาการเป็นไปตาม ประกาศ ก.พ.อ เรื่อง หลักเกณฑ์และวิธีการพิจารณาแต่งตั้งบุคคลให้ดำรงตำแหน่ง ผู้ช่วยศาสตราจารย์ รองศาสตราจารย์ ศาสตราจารย์ พ.ศ. 2561 หรือ พ.ศ. 2563
\end{sphinxadmonition}
\begin{description}
\item[{ระดับที่ ๑}] \leavevmode
มีชื่อ/หัวข้อตำรา/หนังสือ/บทความวิชาการ/บทความวิจัยหรือข้อมูลสำหรับการเขียนผลงานดังกล่าว

\item[{ระดับที่ ๒}] \leavevmode
เป็นไปตามระดับที่ ๑ และจัดทำเค้าโครงของตำรา/หนังสือ/บทความวิชาการ/บทความวิจัยที่ครบถ้วน

\item[{ระดับที่ ๓}] \leavevmode
เป็นไปตามระดับที่ ๒ และจัดทำตำรา/หนังสือ/บทความวิชาการ/บทความวิจัยฉบับร่าง

\item[{ระดับที่ ๔}] \leavevmode
เป็นไปตามระดับที่ ๓ และจัดทำตำรา/หนังสือ/บทความวิชาการ/บทความวิจัยฉบับสมบูรณ์พร้อมเผยแพร่

\item[{ระดับที่ ๕}] \leavevmode
เป็นไปตามระดับที่ ๔ และมีการเผยแพร่ผลงานตำรา/หนังสือ/บความวิชาการ/บทความวิจัยผ่านสื่อสิ่งพิมพ์หรือมิใช่สิ่งพิมพ์

\end{description}

\begin{sphinxadmonition}{note}{Note:}
เอกสารประกอบการสอนหรือ เอกสารคำสอน นับว่าอยู่ในระดับที่ 4
\end{sphinxadmonition}

\begin{sphinxadmonition}{note}{Note:}
ระดับที่ 5 ใช้หลักเกณฑ์การเผยแพร่ตามประกาศ ก.พ.อ เรื่อง หลักเกณฑ์และวิธีการพิจารณาแต่งตั้งบุคคลให้ดำรง ตำแหน่ง ผู้ช่วยศาสตราจารย์
รองศาสตราจารย์ ศาสตราจารย์ พ.ศ. 2561 หรือ พ.ศ. 2563
\end{sphinxadmonition}


\subsection{หลักฐาน}
\label{\detokenize{submission_part1:id10}}\begin{itemize}
\item {} 
ส่งหลักฐานเพียง 1 ผลงานเท่านั้น โดยเป็นหลักฐานที่ตรเองได้ระดับมากที่สุด

\item {} 
กรณีส่งเอกสารประกอบการสอนหรือเอกสารคำสอน \sphinxstylestrong{ต้อง} มีหนังสือบันทึกข้อความเสนอคณบดีเผยแพร่ก่อนเปิดภาคเรียน

\end{itemize}


\bigskip\hrule\bigskip



\section{3. ภาระงานวิจัยและวิชาการอื่น}
\label{\detokenize{submission_part1:id11}}

\subsection{ระดับความสำเร็จในการจัดทำ}
\label{\detokenize{submission_part1:id12}}\begin{description}
\item[{ระดับที่ 1}] \leavevmode
เป็นผู้มีส่วนร่วมโครงการ อย่างน้อย 1 โครงการ

\item[{ระดับที่ 2}] \leavevmode
เป็นผู้มีส่วนร่วมโครงการ 2 โครงการขึ้นไป

\item[{ระดับที่ 3}] \leavevmode
เป็นผู้มีส่วนร่วมในโครงการ หรือมีส่วนร่วมกับหน่วยงานภายนอก
\sphinxstyleemphasis{หมายเหตุ} ผู้มีส่วนร่วมในโครงการ หมายถึง หัวหน้าโครงการ วิทยากร
ผู้ช่วยวิทยากร ผู้รับผิดชอบโครงการ

\item[{ระดับที่ 4}] \leavevmode
บูรณาการงานบริการวิชาการร่วมกับการเรียนการสอน หรืองานวิชาการอื่น

\item[{ระดับที่ 5}] \leavevmode
การบริการทางวิชาการตามระเบียบฯ ว่าด้วยการให้บริการสังคม

\end{description}


\subsection{หลักฐาน}
\label{\detokenize{submission_part1:id13}}\begin{itemize}
\item {} 
ใช้หลักฐานที่เกี่ยวข้องกับงานบริการวิชาการ เช่น
\begin{itemize}
\item {} 
คำสั่งปฏิบัติงาน

\item {} 
หนังสือเชิญเป็นวิทยากร

\item {} 
ภาพถ่าย

\item {} 
ไฟล์อิเล็กทรอนิกส์ (PDF) ที่แสดงถึงการมีส่วนร่วมในโครงการบริการวิชาการภายในและภายนอกหน่วยงาน

\end{itemize}

\item {} 
กรณีคณาจารย์ปฏิบัติหน้าที่หลายหน้าที่ในโครงการบริการวิชาการ ให้ส่งหลักฐานแต่ละหน้าที่ที่ปฏิบัติงานอย่างชัดเจน

\end{itemize}


\bigskip\hrule\bigskip



\section{4. งานทำนุบำรุงศิลปวัฒนธรรม}
\label{\detokenize{submission_part1:id14}}

\subsection{ระดับความสำเร็จในการจัดทำ}
\label{\detokenize{submission_part1:id15}}\begin{description}
\item[{ระดับที่ ๑}] \leavevmode
เป็นผู้มีส่วนร่วมงานทำนุบำรุงศิลปวัฒนธรรมในกิจกรรม/โครงการของหน่วยงาน หรือมหาวิทยาลัย จำนวน 1 กิจกรรม

\item[{ระดับที่ ๒  เป็นผู้มีส่วนร่วมงานทำนุบำรุงศิลปวัฒนธรรมในกิจกรรม/โครงการ}] \leavevmode
เป็นผู้มีส่วนร่วมงานทำนุบำรุงศิลปวัฒนธรรมในกิจกรรม/โครงการของหน่วยงาน หรือมหาวิทยาลัย จำนวน 2 กิจกรรม

\item[{ระดับที่ ๓}] \leavevmode
เป็นผู้มีส่วนร่วมงานทำนุบำรุงศิลปวัฒนธรรมในกิจกรรม/โครงการของหน่วยงาน หรือมหาวิทยาลัย จำนวน 3 กิจกรรม
\sphinxstyleemphasis{หรือ} เป็นคณะกรรมการงานทำนุบำรุงศิลปวัฒนธรรมในกิจกรรม/โครงการของหน่วยงาน หรือมหาวิทยาลัย จำนวน 1 โครงการ

\item[{ระดับที่ ๔}] \leavevmode
เป็นผู้มีส่วนร่วมงานทำนุบำรุงศิลปวัฒนธรรมในกิจกรรม/โครงการของหน่วยงาน หรือมหาวิทยาลัย จำนวน 4 กิจกรรม
\sphinxstyleemphasis{หรือ} เป็นคณะกรรมการงานทำนุบำรุงศิลปวัฒนธรรมในกิจกรรม/โครงการของหน่วยงาน หรือมหาวิทยาลัย จำนวน 2 โครงการ

\item[{ระดับที่ ๕}] \leavevmode
บูรณาการศิลปวัฒนธรรมกับหน่วยงานภายนอก

\end{description}


\subsection{หลักฐาน}
\label{\detokenize{submission_part1:id16}}
ใช้หลักฐานที่เกี่ยวข้องกับงานทำนุบำรุงศิลปวัฒนธรรม เช่น
\begin{itemize}
\item {} 
คำสั่งปฏิบัติงาน

\item {} 
หนังสือขออนุญาตเข้าร่วมงาน/กิจกรรม

\item {} 
ไฟล์อิเล็กทรอนิกส์ (PDF) ที่แสดงถึงงานทำนุบำรุงศิลปวัฒนธรรม

\end{itemize}

เป็นต้น


\section{5. งานอื่น ๆ}
\label{\detokenize{submission_part1:id17}}

\subsection{ระดับความสำเร็จในการจัดทำ}
\label{\detokenize{submission_part1:id18}}\begin{description}
\item[{ระดับที่ 1}] \leavevmode
เข้าร่วมกิจกรรม/งานอื่น ๆ หรืองานที่ได้รับมอบหมายหรือได้รับอนุญาตจากหน่วยงาน หรือมหาวิทยาลัย จำนวน 1 กิจกรรม

\item[{ระดับที่ 2}] \leavevmode
เข้าร่วมกิจกรรม/งานอื่น ๆ หรืองานที่ได้รับมอบหมายหรือได้รับอนุญาตจากหน่วยงาน หรือมหาวิทยาลัย จำนวน 2 กิจกรรม

\item[{ระดับที่ 3}] \leavevmode
เข้าร่วมกิจกรรม/งานอื่น ๆ หรืองานที่ได้รับมอบหมายหรือได้รับอนุญาตจากหน่วยงาน หรือมหาวิทยาลัย จำนวน 3 กิจกรรม

\item[{ระดับที่ 4}] \leavevmode
เข้าร่วมกิจกรรม/งานอื่น ๆ หรืองานที่ได้รับมอบหมายหรือได้รับอนุญาตจากหน่วยงาน หรือมหาวิทยาลัย จำนวน 4 กิจกรรม

\item[{ระดับที่ 5}] \leavevmode
เข้าร่วมกิจกรรม/งานอื่น ๆ หรืองานที่ได้รับมอบหมายหรือได้รับอนุญาตจากหน่วยงาน หรือมหาวิทยาลัย จำนวน 5 กิจกรรมขึ้นไป

\end{description}

\begin{sphinxadmonition}{note}{Note:}
งานอื่น ๆ หรืองานที่ได้รับมอบหมาย หรือได้รับอนุญาตจากหน่วยงานหรือมหาวิทยาลัย
\end{sphinxadmonition}


\subsection{หลักฐาน}
\label{\detokenize{submission_part1:id19}}
ใช้หลักฐานที่เกี่ยวข้องกับงานอื่นๆหรืองานที่ได้รับมอบหมายหรือได้รับอนุญาตที่นอกเหนือจาก งานสอน งานวิจัยและวิชาการอื่น งานบริการวิชาการ และงานทำนุบำรุงศิลปวัฒนธรรม เช่น หนังสือขออนุญาตเข้าร่วมงาน/กิจกรรม หรือ ไฟล์อิเล็กทรอนิกส์ (PDF) ที่แสดงถึงการเข้าร่วมงานอื่นๆหรืองานที่ได้รับมอบหมายหรือได้รับอนุญาต เป็นต้น
\begin{itemize}
\item {} 
แนะนำให้ส่งมากกว่า 5 กิจกรรม

\end{itemize}


\chapter{1. ภาระงานสอน}
\label{\detokenize{1teaching:id1}}\label{\detokenize{1teaching::doc}}\begin{description}
\item[{จำนวนชั่วโมงรวม}] \leavevmode
อย่างน้อย 9 ชั่วโมง ไม่เกิน 25 ชั่วโมง

\end{description}


\section{ภาระงานสอน สำหรับผู้มีตำแหน่ง}
\label{\detokenize{1teaching:id2}}\begin{itemize}
\item {} 
ผู้บริหาร นับว่ามีภาระงานสอน \sphinxhyphen{}\sphinxhyphen{} ชั่วโมงต่อสัปดาห์

\item {} 
หัวหน้าหมวด นับว่ามีภาระงานสอน \sphinxhyphen{}\sphinxhyphen{} ชั่วโมงต่อสัปดาห์

\end{itemize}


\section{คำอธิบายเพิ่มเติม}
\label{\detokenize{1teaching:id3}}

\begin{savenotes}\sphinxattablestart
\centering
\begin{tabular}[t]{|\X{20}{90}|\X{20}{90}|\X{50}{90}|}
\hline
\sphinxstyletheadfamily 
ประเภทห้องเรียน
&\sphinxstyletheadfamily 
ลักษณะงาน
&\sphinxstyletheadfamily 
หมายเหตุ
\\
\hline
หัองเรียน ปวช
&
1.1.3 ก
&\\
\hline
ปรับวุฒิ
&&\\
\hline
สมทบ
&&\\
\hline
ภาคปกติ ภาคเรียนฤดูร้อน
&
ภาระงานภาคปกติ
&\\
\hline
ภาคปกติ ภาคเรียนฤดูร้อน
&
ภาระงานภาคสมทบ
&\\
\hline
\end{tabular}
\par
\sphinxattableend\end{savenotes}

\begin{sphinxadmonition}{note}{Note:}
คณาจารย์ที่สอนเวลาเดียวกัน นับเป็น 1 ห้อง
\end{sphinxadmonition}

\begin{sphinxadmonition}{note}{Note:}
ที่ปรึกษาร่วมโครงงานนักศึกษา นับภาระงานข้อ  \DUrole{xref,std,std-ref}{1\_4\_2}
\end{sphinxadmonition}


\bigskip\hrule\bigskip



\section{1.1 งานสอนนักศึกษาในระดับปริญญาตรี}
\label{\detokenize{1teaching:id4}}

\subsection{1.1.1 การสอนรายวิชาบรรยาย 1 หน่วยกิต}
\label{\detokenize{1teaching:id5}}\begin{description}
\item[{(ก) ภาคปกติ}] \leavevmode
3 ชั่วโมงต่อสัปดาห์

\end{description}

ครอบคลุมถึงภาระงานบรรยาย 1 ชั่วโมง การเตรียมการสอน 1 ชั่วโมง และการตรวจงานนักศึกษา 1 ชั่วโมง
\begin{description}
\item[{(ข) ภาคพิเศษและภาคนอกเวลา::}] \leavevmode
1 ชั่วโมงต่อสัปดาห์

\end{description}


\subsection{1.1.2 การสอนรายวิชาปฏิบัติการ 1 หน่วยกิต}
\label{\detokenize{1teaching:id6}}\begin{description}
\item[{(ก) ภาคปกติ}] \leavevmode
3.5 ชั่วโมงต่อสัปดาห์

\end{description}

ครอบคลุมถึงภาระงานบรรยาย 1 ชั่วโมง การเตรียมการสอน 1 ชั่วโมง และการตรวจงานนักศึกษา 1 ชั่วโมง
\begin{description}
\item[{(ข) ภาคพิเศษและภาคนอกเวลา::}] \leavevmode
1 ชั่วโมงต่อสัปดาห์

\end{description}


\section{1.2 งานสอนนักศึกษาในระดับบัณฑิตศึกษา}
\label{\detokenize{1teaching:id7}}

\section{1.3 งานสอนในระดับต่ำกว่าปริญญาตรี}
\label{\detokenize{1teaching:id8}}

\section{1.4 งานด้านโครงงาน}
\label{\detokenize{1teaching:id9}}
1.4.2 กรรมการสอบโครงงาน


\section{1.5 งานด้านวิทยานิพนธ์และการค้นคว้าอิสระ}
\label{\detokenize{1teaching:id10}}

\section{1.6 การสอนในหลักสูตรอื่นนอกจาก 1.1 \sphinxhyphen{} 1.5}
\label{\detokenize{1teaching:id11}}

\chapter{2. ภาระงานวิจัยและบริการวิชาการ}
\label{\detokenize{2research:id1}}\label{\detokenize{2research::doc}}\begin{description}
\item[{จำนวนชั่วโมงรวม}] \leavevmode
ไม่เกิน 10 ชั่วโมง

\end{description}


\section{เงื่อนไขการได้รับการประเมิน}
\label{\detokenize{2research:id2}}\begin{enumerate}
\sphinxsetlistlabels{\arabic}{enumi}{enumii}{}{.}%
\item {} 
ร่างงานวิจัย หรือ ร่างบทความวิชาการ \sphinxstylestrong{ไม่} สามารถนำมาคิดภาระงานได้

\item {} 
=

\end{enumerate}


\section{สิ่งจำเป็นในหลักฐานการประเมิน}
\label{\detokenize{2research:id3}}\begin{enumerate}
\sphinxsetlistlabels{\arabic}{enumi}{enumii}{}{.}%
\item {} 
\sphinxstylestrong{ต้อง} คำนวณและใส่คะแนนภาระงาน ตามประกาศทางคณะวิทยาศาสตร์และเทคโนโลยี ถ้าไม่คำนวณจะขอ

\item {} 
เนื่องจากผลงานใน {\hyperref[\detokenize{2research:id8}]{\sphinxcrossref{\DUrole{std,std-ref}{2.3}}}} สามารถเคลมได้ 2 รอบ ฉะนั้นต้องระบุเพิ่มเติมด้วยว่าเป็น \sphinxstyleemphasis{ครั้งที่ 1} หรือ \sphinxstyleemphasis{ครั้งที่ 2}

\item {} 
ระบุวันเวลาอย่างชัดเจน

\end{enumerate}


\bigskip\hrule\bigskip



\section{2.1 การร่วมทำวิจัย}
\label{\detokenize{2research:id4}}
ไม่นับโครงการที่มีการขยายระยะเวลา


\subsection{2.1.1 มีส่วนร่วมในโครงการวิจัยตั้งแต่ร้อยละ 60 ขึ้นไป}
\label{\detokenize{2research:id5}}
3.5 ชั่วโมงต่อสัปดาห์


\subsection{2.1.2 มีส่วนร่วมในโครงการวิจัยตั้งแต่ร้อยละ 40 \sphinxhyphen{} 59}
\label{\detokenize{2research:id6}}
2.5 ชั่วโมงต่อสัปดาห์


\section{2.2 ผู้อำนวยแผนวิจัย}
\label{\detokenize{2research:id7}}
2 ชั่วโมงต่อสัปดาห์


\section{2.3 การตีพิมพ์เผยแพร่บทความวิจัย}
\label{\detokenize{2research:id8}}\label{\detokenize{2research:id9}}
\begin{sphinxadmonition}{important}{Important:}
ผลงานในส่วนนี้ สามารถเคลมได้ 2 รอบการประเมินติดต่อกัน
\end{sphinxadmonition}


\subsection{2.3.1 การเข้าร่วมประชุมหรือสัมมนาทางวิชาการ ที่ผ่านการอนุมัติหรือเห็นชอบจากหน่วยงาน}
\label{\detokenize{2research:id10}}
0.5 ชั่วโมงต่อสัปดาห์

\begin{sphinxadmonition}{warning}{Warning:}
ต้องมีการนำเสนอผลงาน หรือถ้าเป็นอาจารย์ที่ปรึกษาโครงการวิจัยให้นักศึกษาที่ไปนำเสนอในงานประชุมวิชาการ จะต้องมีชื่ออยู่ในการนำเสนอนั้นด้วย
\end{sphinxadmonition}


\section{2.4 การเผยแพร่ผลงานสร้างสรรค์}
\label{\detokenize{2research:id11}}

\section{2.5 การนำผลงานวิจัยไปใช้ประโยชน์}
\label{\detokenize{2research:id12}}
การนำผลงานวิจัยไปใช้ประโยชน์นอกมหาวิทยาลัย ในส่วนราชการระดับกรม หรือเทียบเท่าขึ้นไป....


\chapter{3. ภาระงานวิจัยและบริการวิชาการ}
\label{\detokenize{3service:id1}}\label{\detokenize{3service::doc}}
ภาระงานวิจัยและบริการวิชาการรวมกันไม่เกิน 10 ชั่วโมง


\section{การพิจารณาสัดส่วน}
\label{\detokenize{3service:id2}}
การคิดภาระงานบริการวิชาการ ให้พิจารณาตามสัดส่วนของการมีส่วนร่วมในงานบริการทางวิชาการนั้น ๆ โดยต้องมีเอกสารยืนยันการมีสัดส่วนผลงานจากผู้มีส่วนร่วมทุกคน

ผู้รับผิดชอบในโครงการบริการวิชาการคิดคำนวณคะแนนภาระงานให้ผู้ปฏิบัติงานในโครงการทุกคน โดยให้ผู้ที่รับผิดชอบงานแบบเดียวกันได้รับคะแนนภาระงานที่เท่ากัน

คิดภาระงานด้วยสูตรดังต่อไปนี้
\begin{quote}

(ชั่วโมงที่ปฏิบัติงานจริง) x (อัตราส่วนของหน้าที่) ÷ 15
\end{quote}

โดยที่
\begin{itemize}
\item {} 
ชั่วโมงที่ปฏิบัติงานจริง ไม่นับเวลาพักกลางวันและเวลาเตรียมงาน

\item {} 
อัตราส่วนของหน้าที่ ระบุไว้ด้านล่าง

\item {} 
การหารด้วย 15 แสดงถึงการเฉลี่ยภาระงานนั้นใน 1 ภาคเรียน (15 สัปดาห์)

\end{itemize}

เมื่อคำนวณเสร็จแล้ว ให้ปัดเศษให้เป็นทศนิยม 2 ตำแหน่ง และถ้ามีมากกว่า 1 หน้าที่ในโครงการ ให้คิดภาระงานแยกตามหน้าที่

ตัวอย่างเช่น
\begin{itemize}
\item {} 
ผู้รับผิดชอบโครงการ เวลา 8:30 \sphinxhyphen{} 16:30 นับเป็น 8 ชั่วโมง หักเวลาพักกลางวัน 1 ชั่วโมง เหลือ 7 ชั่วโมง ฉะนั้นจะได้
\begin{quote}

7 x 0.7 ÷ 15 = 3.27
\end{quote}

\item {} 
ผู้ร่วมโครงการ เวลา 8:30 \sphinxhyphen{} 12:30 จะได้
\begin{quote}

4 x 0.3 ÷ 15 = 0.8
\end{quote}

\end{itemize}


\section{เงื่อนไขการได้รับการประเมิน}
\label{\detokenize{3service:id3}}\begin{enumerate}
\sphinxsetlistlabels{\arabic}{enumi}{enumii}{}{.}%
\item {} 
กรณีคณาจารย์ปฏิบัติหน้าที่หลายหน้าที่ในโครงการบริการวิชาการ ให้ส่งหลักฐานแต่ละหน้าที่ที่ปฏิบัติงานอย่างชัดเจน

\item {} 
กรณีเป็นการบริการวิชาการเชิงพาณิชย์ให้แนบใบเสร็จหรือหลักฐานการนำส่งเงินเข้ามหาวิทยาลัยเทคโนโลราชมงคลพระนครอย่างชัดเจน

\item {} 
กรณีระยะเวลาโครงการบริการสังคมมีการปฏิบัติงานถึง 2 รอบการประเมิน สามารถนับได้ 2 รอบการประเมิน โดยต้องระบุเพิ่มเติมด้วยว่าเป็น \sphinxstyleemphasis{ครั้งที่ 1} หรือ \sphinxstyleemphasis{ครั้งที่ 2}

\end{enumerate}


\section{สิ่งจำเป็นในหลักฐานการประเมิน}
\label{\detokenize{3service:id4}}\begin{itemize}
\item {} 
\sphinxstylestrong{ต้อง} มีหลักฐานที่ระบุเวลาการปฏิบัติงาน ถ้าไม่มีหลักฐานหรือหลักฐานไม่ได้ระบุเวลา คณะวิทยาศาสตร์และเทคโนโลยีขอสงวนสิทธิ์ในการดำเนินการ

\item {} 
คณาจารย์ที่ดำเนินการไปบริการวิชาการภายนอกหน่วยงานให้แนบตารางกำหนดการบอกเวลาปฏิบัติงานอย่างชัดเจน

\item {} 
การได้รับเชิญเป็นวิทยากรให้ส่งหลักฐานการไปปฏิบัติหน้าที่อย่างชัดเจน เช่น ภาพถ่าย

\item {} 
ให้คณาจารย์เรียงหลักฐานตามแบบรายงานภาระงานให้มีความสอดคล้องกัน และ/หรือ ระบุหมายเลขของหลักฐานให้มีความชัดเจน

\end{itemize}


\bigskip\hrule\bigskip



\section{3.1 การบริการวิชาการแก่สังคม}
\label{\detokenize{3service:id5}}

\subsection{3.1.1 ปฏิบัติโครงการบริการวิชาการแก่สังคมตามแผน}
\label{\detokenize{3service:id6}}

\subsubsection{3.1.1.1 ผู้ร่วมกิจกรรมในโครงการ}
\label{\detokenize{3service:id7}}\begin{description}
\item[{(ก) ผู้รับผิดชอบโครงการ}] \leavevmode
ร้อยละ 70

\item[{(ข) กรรมการหรือผู้ร่วมโครงการ}] \leavevmode
ร้อยละ 30

\end{description}


\subsubsection{3.1.1.2 วิทยากร}
\label{\detokenize{3service:id8}}\begin{description}
\item[{(ก) วิทยากร}] \leavevmode
ร้อยละ 70

\item[{(ข) ผู้ช่วยวิทยากร}] \leavevmode
ร้อยละ 70

\end{description}


\section{3.2 การบริการวิชาการเชิงพาณิชย์ ที่มีการเซ็นสัญญาที่หน่วยงาน หรือมหาวิทยาลัย หรือมีเอกสารยืนยันเป็นลายลักษณ์อักษร}
\label{\detokenize{3service:id9}}

\chapter{4. ภาระงานบำรุงศิลปวัฒนธรรม}
\label{\detokenize{4culture:id1}}\label{\detokenize{4culture::doc}}\begin{description}
\item[{นิยาม}] \leavevmode
งานบำรุงศิลปวัฒนธรรม คืองานที่ ........

\item[{จำนวนชั่วโมงรวม}] \leavevmode
ไม่เกิน 5 ชั่วโมงต่อสัปดาห์

\end{description}


\bigskip\hrule\bigskip



\section{เงื่อนไขการได้รับการประเมิน}
\label{\detokenize{4culture:id2}}\begin{enumerate}
\sphinxsetlistlabels{\arabic}{enumi}{enumii}{}{.}%
\item {} 
งานที่คณะวิทยาศาสตร์และเทคโนโลยี และ/หรือ ที่มหาวิทยาลัยเทคโนโลยีราชมงคลพระนครดำเนินการจัดขึ้น (ได้แก่ {\hyperref[\detokenize{4culture:id4}]{\sphinxcrossref{\DUrole{std,std-ref}{4.1}}}} และ {\hyperref[\detokenize{4culture:id13}]{\sphinxcrossref{\DUrole{std,std-ref}{4.5}}}})
\begin{itemize}
\item {} 
ไม่ต้องทำบันทึกข้อความ

\end{itemize}

\item {} 
งานจากหน่วยงานภายนอก
\begin{itemize}
\item {} 
นำมาคิดภาระงานได้ไม่เกิน 2 ครั้ง ต่อรอบการประเมิน

\item {} 
ก่อนวันงาน ต้องจัดทำบันทึกข้อความขออนุญาตจากหัวหน้าหน่วยงาน

\item {} 
หลังวันงาน ต้องจัดทำบันทึกข้อความรายงานผลการดำเนินงานต่อหัวหน้าหน่วยงาน

\item {} 
ถ้าเข้าร่วมโดยยังไม่ได้รับอนุญาตจากหัวหน้าหน่วยงาน คณาจารย์สามารถนำผลการเข้าร่วมงานดังกล่าวมาคิดภาระงานได้ไม่เกิน 2 ครั้ง ต่อรอบการประเมิน ทั้งนี้ภายหลังการเข้าร่วมงานให้คณาจารย์เร่งจัดทำบันทึกข้อความรายงานหัวหน้าหน่วยงานโดยทันที

\end{itemize}

\item {} 
วันสำคัญของชาติ วันสำคัญทางศาสนา เทศกาลตามประเพณี และเทศกาลตามที่รัฐบาลประกาศ สามารถนำมาคิดภาระงานได้ 1 ครั้ง ต่อวันสำคัญ และ/หรือ เทศกาลนั้นๆ (ถึงแม้ว่างาน และ/หรือ กิจกรรมดังกล่าวข้างต้นจะมีการจัดงาน และ/หรือ กิจกรรมหลายครั้งหรือหลายวันในวันสำคัญ และ/หรือ เทศกาลนั้น)

\end{enumerate}


\section{สิ่งจำเป็นในหลักฐานการประเมิน}
\label{\detokenize{4culture:id3}}\begin{enumerate}
\sphinxsetlistlabels{\arabic}{enumi}{enumii}{}{.}%
\item {} 
ประกาศจากมหาวิทยาลัย

\item {} 
ระบุวันเวลาอย่างชัดเจน

\end{enumerate}


\bigskip\hrule\bigskip



\section{4.1 การเข้าร่วมในกิจกรรม/โครงการทำนุบำรุงศิลปวัฒนธรรมของมหาวิทยาลัย หรือหน่วยงานภายนอก}
\label{\detokenize{4culture:id4}}\label{\detokenize{4culture:id5}}\begin{description}
\item[{การคำนวณภาระงาน}] \leavevmode
0.5 ชั่วโมงต่อสัปดาห์

\end{description}


\subsection{ประเภทงาน}
\label{\detokenize{4culture:id6}}\begin{description}
\item[{ทำบุญด้วยตนเอง}] \leavevmode
นำมาคิดภาระงานได้ไม่เกิน 6 ครั้ง ต่อรอบการประเมิน โดยไม่นับรวมกับหน่วยงานภายนอก

\end{description}


\bigskip\hrule\bigskip



\section{4.2 การเป็นผู้รับผิดชอบในกิจกรรม/โครงการทำนุบำรุงศิลปวัฒนธรรมภายในประเทศ}
\label{\detokenize{4culture:id7}}\label{\detokenize{4culture:id8}}\begin{description}
\item[{การคำนวณภาระงาน}] \leavevmode
1 ชั่วโมงต่อสัปดาห์

\end{description}


\bigskip\hrule\bigskip



\section{4.3 การเข้าร่วมในกิจกรรม/โครงการทำนุบำรุงศิลปวัฒนธรรมภายนอกประเทศ}
\label{\detokenize{4culture:id9}}\label{\detokenize{4culture:id10}}\begin{description}
\item[{การคำนวณภาระงาน}] \leavevmode
1 ชั่วโมงต่อสัปดาห์

\end{description}


\bigskip\hrule\bigskip



\section{4.4 การเป็นผู้รับผิดชอบในกิจกรรม/โครงการทำนุบำรุงศิลปวัฒนธรรมภายนอกประเทศ}
\label{\detokenize{4culture:id11}}\label{\detokenize{4culture:id12}}\begin{description}
\item[{การคำนวณภาระงาน}] \leavevmode
1.5 ชั่วโมงต่อสัปดาห์

\end{description}


\bigskip\hrule\bigskip



\section{4.5 การเข้าร่วมในกิจกรรม/โครงการทำนุบำรุงศิลปวัฒนธรรมของคณะวิทยาศาสตร์และเทคโนโลยี}
\label{\detokenize{4culture:id13}}\label{\detokenize{4culture:id14}}\begin{description}
\item[{การคำนวณภาระงาน}] \leavevmode
0.5 ชั่วโมงต่อสัปดาห์

\end{description}


\bigskip\hrule\bigskip



\chapter{5. ภาระงานอื่น ๆ}
\label{\detokenize{5etc:id1}}\label{\detokenize{5etc::doc}}\begin{description}
\item[{จำนวนชั่วโมงรวม}] \leavevmode
ไม่เกิน 10 ชั่วโมง

\end{description}

งานอื่น ๆ คืองานที่นอกเหนือจาก งานสอน งานวิจัยและวิชาการอื่น งานบริการวิชาการ และงานทำนุบำรุงศิลปวัฒนธรรม


\section{เงื่อนไขการได้รับการประเมิน}
\label{\detokenize{5etc:id2}}\begin{enumerate}
\sphinxsetlistlabels{\arabic}{enumi}{enumii}{}{.}%
\item {} 
ร่างงานวิจัย หรือ ร่างบทความวิชาการ \sphinxstylestrong{ไม่} สามารถนำมาคิดภาระงานได้

\item {} 
=

\end{enumerate}


\section{สิ่งจำเป็นในหลักฐานการประเมิน}
\label{\detokenize{5etc:id3}}\begin{enumerate}
\sphinxsetlistlabels{\arabic}{enumi}{enumii}{}{.}%
\item {} 
\sphinxstylestrong{ต้อง} ระบุวันเวลาอย่างชัดเจน คณะวิทยาศาสตร์และเทคโนโลยีขอสงวนสิทธิ์ในการดำเนินการ

\item {} 
ทำบันทึกข้อความขออนุญาติเข้าร่วมงานหรือกิจกรรมล่วงหน้า

\end{enumerate}


\section{5.1 งานบริการจัดการสอนมากกว่า 1 ศูนย์การศึกษา และ/หรือ นอกศูนย์ที่ตั้งของคณะ}
\label{\detokenize{5etc:id4}}\begin{description}
\item[{(ก) การสอนนอกศูนย์ที่ตั้งของคณะ}] \leavevmode
1 ชั่วโมงต่อสัปดาห์ (ไม่รวมการคุมสอบต่างศูนย์การศึกษา)

\item[{(ข) 2 ศูนย์}] \leavevmode
2 ชั่วโมงต่อสัปดาห์

\item[{(ค) 3 ศูนย์}] \leavevmode
3 ชั่วโมงต่อสัปดาห์

\item[{(ง) 4 ศูนย์}] \leavevmode
4 ชั่วโมงต่อสัปดาห์

\item[{(จ) การคุมสอบต่างศูนย์}] \leavevmode
0.5 ชั่วโมงต่อสัปดาห์

\end{description}


\section{5.2 งานพัฒนานักศึกษา}
\label{\detokenize{5etc:id5}}

\section{5.3 งานพัฒนาองค์กร}
\label{\detokenize{5etc:id6}}\begin{description}
\item[{(จ) ภาระงานตาม KPI ของคณะ}] \leavevmode
1 ชั่วโมงต่อสัปดาห์

\end{description}

\begin{sphinxadmonition}{note}{Note:}
ผู้ดำรงตำแหน่งทางวิชาการ และผู้มีคุณวุฒิ ปริญญาเอก นับเป็นภาระงานตาม KPI สามารถนับภาระงานได้ \sphinxstylestrong{ทุกรอบ} การประเมิน
\end{sphinxadmonition}


\section{5.4 งานพัฒนาตนเอง}
\label{\detokenize{5etc:id7}}

\section{5.5 งานจิตอาสา}
\label{\detokenize{5etc:id8}}

\chapter{การนำส่งหลักฐานการประเมิน (สายวิชาการ) องค์ 2}
\label{\detokenize{submission_part2:id1}}\label{\detokenize{submission_part2::doc}}

\section{วิธีการจัดเตรียมเอกสาร}
\label{\detokenize{submission_part2:id2}}

\subsection{ฉบับเอกสาร}
\label{\detokenize{submission_part2:id3}}\begin{itemize}
\item {} 
(แนะนำ) ไฮไลท์ข้อความดังนี้
\begin{itemize}
\item {} 
ชื่ออาจารย์

\item {} 
ชื่องาน

\item {} 
วันและเวลา

\end{itemize}

\end{itemize}


\subsection{ฉบับไฟล์}
\label{\detokenize{submission_part2:id4}}

\subsubsection{ประเภทไฟล์}
\label{\detokenize{submission_part2:id5}}\begin{itemize}
\item {} 
แนะนำให้ใช้ .pdf

\item {} 
.doc

\item {} 
ไฟล์ภาพสกุล .png, .jpeg, .gif, ฯลฯ ให้หมุนภาพตั้งขึ้นเพื่อให้อ่านง่าย

\item {} 
(แนะนำ) ไฮไลท์ข้อความดังนี้
\begin{itemize}
\item {} 
ชื่ออาจารย์

\item {} 
ชื่องาน

\item {} 
วันและเวลา

\end{itemize}

\item {} 
ตั้งชื่อไฟล์ดังนี้
\begin{quote}

1\_1G
\end{quote}

\end{itemize}

โดยที่ 1\_1 แสดงถึง และ G เป็นการเรียงลำดับไฟล์


\chapter{อภิทานศัพท์}
\label{\detokenize{glossary:id1}}\label{\detokenize{glossary::doc}}\begin{description}
\item[{เอกสารประกอบการสอน\index{เอกสารประกอบการสอน@\spxentry{เอกสารประกอบการสอน}|spxpagem}\phantomsection\label{\detokenize{glossary:term-0}}}] \leavevmode
เอกสารที่ใช้ประกอบการสอน

\item[{หนังสือ\index{หนังสือ@\spxentry{หนังสือ}|spxpagem}\phantomsection\label{\detokenize{glossary:term-1}}}] \leavevmode
หนังสือที่ต้องตีพิมพ์

\item[{ทำบุญ\index{ทำบุญ@\spxentry{ทำบุญ}|spxpagem}\phantomsection\label{\detokenize{glossary:term-2}}}] \leavevmode
ทำดีได้ดี

\end{description}


\chapter{คำถามที่พบบ่อย}
\label{\detokenize{faq:id1}}\label{\detokenize{faq::doc}}
\begin{sphinxadmonition}{note}{Note:}
Q: ทำไมเยอะจัง
A: นั่นสิ
\end{sphinxadmonition}

ประกาศ ก.พ.อ เรื่อง หลักเกณฑ์และวิธีการพิจารณาแต่งตั้งบุคคลให้ดำรงตำแหน่ง ผู้ช่วยศาสตราจารย์ รองศาสตราจารย์ ศาสตราจารย์

\sphinxcode{\sphinxupquote{pdf}}



\renewcommand{\indexname}{Index}
\printindex
\end{document}