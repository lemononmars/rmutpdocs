%% Generated by Sphinx.
\def\sphinxdocclass{report}
\documentclass[a4paper,12pt,english]{sphinxmanual}
\ifdefined\pdfpxdimen
   \let\sphinxpxdimen\pdfpxdimen\else\newdimen\sphinxpxdimen
\fi \sphinxpxdimen=.75bp\relax
%% turn off hyperref patch of \index as sphinx.xdy xindy module takes care of
%% suitable \hyperpage mark-up, working around hyperref-xindy incompatibility
\PassOptionsToPackage{hyperindex=false}{hyperref}

\PassOptionsToPackage{warn}{textcomp}

\catcode`^^^^00a0\active\protected\def^^^^00a0{\leavevmode\nobreak\ }
\usepackage{cmap}
\usepackage{fontspec}
\defaultfontfeatures[\rmfamily,\sffamily,\ttfamily]{}
\usepackage{amsmath,amssymb,amstext}
\usepackage{polyglossia}
\setmainlanguage{english}



\setmainfont{TeX Gyre Termes}
\setsansfont{TeX Gyre Heros}
\setmonofont{TeX Gyre Cursor}
    

\usepackage[Bjornstrup]{fncychap}
\usepackage{sphinx}

\fvset{fontsize=\small}
\usepackage{geometry}


% Include hyperref last.
\usepackage{hyperref}
% Fix anchor placement for figures with captions.
\usepackage{hypcap}% it must be loaded after hyperref.
% Set up styles of URL: it should be placed after hyperref.
\urlstyle{same}

\addto\captionsenglish{\renewcommand{\contentsname}{สารบัญ}}

\usepackage{sphinxmessages}
\setcounter{tocdepth}{1}


\XeTeXlinebreaklocale ”th”
\XeTeXlinebreakskip = 0pt plus 0pt

\usepackage{fontspec}
\defaultfontfeatures{Mapping=tex-text}

\newfontfamily{\thaifont}[Scale=MatchUppercase,Mapping=textext]{TH Sarabun New}
\newenvironment{thailang}{\thaifont}{}
\usepackage[Latin,Thai]{ucharclasses}

\setTransitionTo{Thai}{\begin{thailang}}
\setTransitionFrom{Thai}{\end{thailang}}

\usepackage{setspace}

\usepackage{polyglossia}
\setdefaultlanguage{english}
\setotherlanguage{thai}

\AtBeginDocument\captionsthai % Force the caption to Thai
    

\title{ภาระงานบุคลากรสายวิชาการ}
\date{พ.ย. 10, 2020}
\release{}
\author{Sakulbuth Ekvittayaniphon}
\newcommand{\sphinxlogo}{\vbox{}}
\renewcommand{\releasename}{}
\makeindex
\begin{document}

\pagestyle{empty}
\sphinxmaketitle
\pagestyle{plain}
\sphinxtableofcontents
\pagestyle{normal}
\phantomsection\label{\detokenize{index::doc}}


คณะวิทยาศาสตร์และเทคโนโลยี มหาวิทยาลัยเทคโนโลยีราชมงคลพระนคร


\chapter{วิธีใช้เวบไซท์}
\label{\detokenize{index:id2}}\begin{itemize}
\item {} 
กดที่แถบด้านขวาเพื่อเลือกหัวข้อ

\item {} 
ข้อความที่อยู่ในกล่องสีเทาคือตัวอย่าง

\item {} 
ข้อความที่ \sphinxstylestrong{เน้น} คือข้อความสำคัญ

\end{itemize}

\begin{sphinxadmonition}{warning}{Warning:}
ระวังนะ!
\end{sphinxadmonition}


\section{ที่มา}
\label{\detokenize{index:id3}}
ติดต่อพวกเรา


\subsection{1. ภาระงานสอน}
\label{\detokenize{1teaching:id1}}\label{\detokenize{1teaching::doc}}
ภาระงานสอนอย่างน้อย 9 ชั่วโมง ไม่เกิน 25 ชั่วโมง


\subsubsection{1.1 งานสอนนักศึกษาในระดับปริญญาตรี}
\label{\detokenize{1teaching:id2}}

\paragraph{1.1.1 การสอนรายวิชาบรรยาย 1 หน่วยกิต}
\label{\detokenize{1teaching:id3}}\begin{description}
\item[{(ก) ภาคปกติ: ::}] \leavevmode
3 ชั่วโมงต่อสัปดาห์

\end{description}
\begin{itemize}\setlength{\itemsep}{0pt}\setlength{\parskip}{0pt}
\item {} 
(ก) ภาคปกติ

\item {} 
3 ชั่วโมงต่อสัปดาห์

\item {} 
อะไรสักอย่าง

\end{itemize}

\begin{sphinxadmonition}{note}{Note:}
ครอบคลุมถึงภาระงานบรรยาย 1 ชั่วโมง การเตรียมการสอน 1 ชั่วโมง และการตรวจงานนักศึกษา 1 ชั่วโมง
\end{sphinxadmonition}

\begin{sphinxadmonition}{warning}{Warning:}
ปวช ไม่นับเป็นปริญญาตรี
\end{sphinxadmonition}


\sphinxstrong{See also:}



\begin{savenotes}\sphinxattablestart
\centering
\begin{tabulary}{\linewidth}[t]{|T|T|}
\hline
\sphinxstyletheadfamily 
A
&\sphinxstyletheadfamily 
not A
\\
\hline
False
&
True
\\
\hline
True
&
False
\\
\hline
True
&
False
\\
\hline
\end{tabulary}
\par
\sphinxattableend\end{savenotes}




\begin{savenotes}\sphinxattablestart
\centering
\sphinxcapstartof{table}
\sphinxthecaptionisattop
\sphinxcaption{Frozen Delights!}\label{\detokenize{1teaching:id9}}
\sphinxaftertopcaption
\begin{tabular}[t]{|\X{15}{55}|\X{10}{55}|\X{30}{55}|}
\hline
\sphinxstyletheadfamily 
Treat
&\sphinxstyletheadfamily 
Quantity
&\sphinxstyletheadfamily 
Description
\\
\hline
Albatross
&
2.99
&
On a stick!
\\
\hline
Crunchy Frog
&
1.49
&
If we took the bones out, it wouldn't be
crunchy, now would it?
\\
\hline
Gannet Ripple
&
1.99
&
On a stick!
\\
\hline
\end{tabular}
\par
\sphinxattableend\end{savenotes}

read :term:'testt'

\sphinxstyleabbreviation{LIFO} (last\sphinxhyphen{}in, first\sphinxhyphen{}out).
\begin{equation}\label{equation:1teaching:euler}
\begin{split}e^{i\pi} + 1 = 0\end{split}
\end{equation}
Euler's identity, equation \eqref{equation:1teaching:euler}, was elected one of the
most beautiful mathematical formulas.
\begin{description}
\item[{(ข) ภาคพิเศษและภาคนอกเวลา::}] \leavevmode
1 ชั่วโมงต่อสัปดาห์

\end{description}
\begin{description}
\item[{test\index{test@\spxentry{test}|spxpagem}\phantomsection\label{\detokenize{1teaching:term-test}}}] \leavevmode
ตีพิมพ์

\item[{testt\index{testt@\spxentry{testt}|spxpagem}\phantomsection\label{\detokenize{1teaching:term-testt}}}] \leavevmode
ไม่ต้องมั้ง

\end{description}


\subsubsection{1.2 งานสอนนักศึกษาในระดับบัณฑิตศึกษา}
\label{\detokenize{1teaching:id4}}

\subsubsection{1.3 งานสอนในระดับต่ำกว่าปริญญาตรี}
\label{\detokenize{1teaching:id5}}

\subsubsection{1.4 งานด้านโครงงาน}
\label{\detokenize{1teaching:id6}}

\subsubsection{1.5 งานด้านวิทยานิพนธ์และการค้นคว้าอิสระ}
\label{\detokenize{1teaching:id7}}

\subsubsection{1.6 การสอนในหลักสูตรอื่นนอกจาก 1.1 \sphinxhyphen{} 1.5}
\label{\detokenize{1teaching:id8}}

\subsection{2. ภาระงานวิจัยและบริการวิชาการ}
\label{\detokenize{2research:id1}}\label{\detokenize{2research::doc}}
ภาระงานวิจัยและบริการวิชาการรวมกันไม่เกิน 10 ชั่วโมง


\subsubsection{2.1 การร่วมทำวิจัย}
\label{\detokenize{2research:id2}}

\paragraph{2.1.1 มีส่วนร่วมในโครงการวิจัยตั้งแต่ร้อยละ 60 ขึ้นไป}
\label{\detokenize{2research:id3}}\begin{description}
\item[{::}] \leavevmode
3.5 \sphinxcode{\sphinxupquote{ชั่วโมงต่อสัปดาห์}} {\color{red}\bfseries{}|hpw|}

\end{description}


\paragraph{2.2 งานบริการวิชาการ}
\label{\detokenize{2research:id4}}

\subsection{2. ภาระงานวิจัยและบริการวิชาการ}
\label{\detokenize{3service:id1}}\label{\detokenize{3service::doc}}
ภาระงานวิจัยและบริการวิชาการรวมกันไม่เกิน 10 ชั่วโมง


\subsubsection{2.1 การร่วมทำวิจัย}
\label{\detokenize{3service:id2}}

\paragraph{2.1.1 มีส่วนร่วมในโครงการวิจัยตั้งแต่ร้อยละ 60 ขึ้นไป}
\label{\detokenize{3service:id3}}\begin{description}
\item[{::}] \leavevmode
3.5 \sphinxcode{\sphinxupquote{ชั่วโมงต่อสัปดาห์}} {\color{red}\bfseries{}|hpw|}

\end{description}


\paragraph{2.2 งานบริการวิชาการ}
\label{\detokenize{3service:id4}}

\subsection{2. ภาระงานวิจัยและบริการวิชาการ}
\label{\detokenize{4culture:id1}}\label{\detokenize{4culture::doc}}
ภาระงานวิจัยและบริการวิชาการรวมกันไม่เกิน 10 ชั่วโมง


\subsubsection{2.1 การร่วมทำวิจัย}
\label{\detokenize{4culture:id2}}

\paragraph{2.1.1 มีส่วนร่วมในโครงการวิจัยตั้งแต่ร้อยละ 60 ขึ้นไป}
\label{\detokenize{4culture:id3}}\begin{description}
\item[{::}] \leavevmode
3.5 \sphinxcode{\sphinxupquote{ชั่วโมงต่อสัปดาห์}} {\color{red}\bfseries{}|hpw|}

\end{description}


\paragraph{2.2 งานบริการวิชาการ}
\label{\detokenize{4culture:id4}}

\subsection{2. ภาระงานวิจัยและบริการวิชาการ}
\label{\detokenize{5etc:id1}}\label{\detokenize{5etc::doc}}
ภาระงานวิจัยและบริการวิชาการรวมกันไม่เกิน 10 ชั่วโมง


\subsubsection{2.1 การร่วมทำวิจัย}
\label{\detokenize{5etc:id2}}

\paragraph{2.1.1 มีส่วนร่วมในโครงการวิจัยตั้งแต่ร้อยละ 60 ขึ้นไป}
\label{\detokenize{5etc:id3}}\begin{description}
\item[{::}] \leavevmode
3.5 \sphinxcode{\sphinxupquote{ชั่วโมงต่อสัปดาห์}}

\end{description}


\paragraph{2.2 งานบริการวิชาการ}
\label{\detokenize{5etc:id4}}

\subsection{อภิทานศัพท์}
\label{\detokenize{glossary:id1}}\label{\detokenize{glossary::doc}}\begin{description}
\item[{เอกสารประกอบการสอน\index{เอกสารประกอบการสอน@\spxentry{เอกสารประกอบการสอน}|spxpagem}\phantomsection\label{\detokenize{glossary:term-0}}}] \leavevmode
เอกสารที่ใช้ประกอบการสอน

\item[{หนังสือ\index{หนังสือ@\spxentry{หนังสือ}|spxpagem}\phantomsection\label{\detokenize{glossary:term-1}}}] \leavevmode
หนังสือที่ต้องตีพิมพ์

\item[{ทำบุญ\index{ทำบุญ@\spxentry{ทำบุญ}|spxpagem}\phantomsection\label{\detokenize{glossary:term-2}}}] \leavevmode
ทำดีได้ดี

\end{description}


\subsection{คำถามที่พบบ่อย}
\label{\detokenize{faq:id1}}\label{\detokenize{faq::doc}}


\renewcommand{\indexname}{Index}
\printindex
\end{document}