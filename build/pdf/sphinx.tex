%% Generated by Sphinx.
\def\sphinxdocclass{report}
\documentclass[a4paper,12pt,english]{sphinxmanual}
\ifdefined\pdfpxdimen
   \let\sphinxpxdimen\pdfpxdimen\else\newdimen\sphinxpxdimen
\fi \sphinxpxdimen=.75bp\relax
%% turn off hyperref patch of \index as sphinx.xdy xindy module takes care of
%% suitable \hyperpage mark-up, working around hyperref-xindy incompatibility
\PassOptionsToPackage{hyperindex=false}{hyperref}

\PassOptionsToPackage{warn}{textcomp}

\catcode`^^^^00a0\active\protected\def^^^^00a0{\leavevmode\nobreak\ }
\usepackage{cmap}
\usepackage{fontspec}
\defaultfontfeatures[\rmfamily,\sffamily,\ttfamily]{}
\usepackage{amsmath,amssymb,amstext}
\usepackage{polyglossia}
\setmainlanguage{english}



\setmainfont{TeX Gyre Termes}
\setsansfont{TeX Gyre Heros}
\setmonofont{TeX Gyre Cursor}
    

\usepackage[Bjornstrup]{fncychap}
\usepackage{sphinx}

\fvset{fontsize=\small}
\usepackage{geometry}


% Include hyperref last.
\usepackage{hyperref}
% Fix anchor placement for figures with captions.
\usepackage{hypcap}% it must be loaded after hyperref.
% Set up styles of URL: it should be placed after hyperref.
\urlstyle{same}

\addto\captionsenglish{\renewcommand{\contentsname}{สารบัญ}}

\usepackage{sphinxmessages}
\setcounter{tocdepth}{0}



\setcounter{tocdepth}{1}

\usepackage{titlesec}
\usepackage{tocloft}
\renewcommand\numberline[1]{}

\renewcommand{\chaptername}{}
\renewcommand{\thechapter}{}

\titleformat{\section}
{\normalfont%
\vspace{6pt}%
\sffamily\bfseries\Large}
{}
{.5em}
{}

\titleformat{\subsection}
{\normalfont}
{}
{8pt}
{\Large\bfseries}

\XeTeXlinebreaklocale ”th”
\XeTeXlinebreakskip = 0pt plus 0pt

\usepackage{fontspec}
\defaultfontfeatures{Mapping=tex-text}

\newfontfamily{\thaifont}[Scale=MatchUppercase,Mapping=textext]{TH Sarabun New}
\newenvironment{thailang}{\thaifont}{}
\usepackage[Latin,Thai]{ucharclasses}

\setTransitionTo{Thai}{\begin{thailang}}
\setTransitionFrom{Thai}{\end{thailang}}

\usepackage{setspace}

\usepackage{polyglossia}
\setdefaultlanguage{english}
\setotherlanguage{thai}

\AtBeginDocument\captionsthai
    

\title{ภาระงานบุคลากรสายวิชาการ}
\date{มี.ค. 08, 2021}
\release{}
\author{คณะวิทยาศาสตร์และเทคโนโลยี \textbackslash{} มหาวิทยาลัยราชมงคลพระนคร}
\newcommand{\sphinxlogo}{\vbox{}}
\renewcommand{\releasename}{}
\makeindex
\begin{document}

\pagestyle{empty}
\sphinxmaketitle
\pagestyle{plain}
\sphinxtableofcontents
\pagestyle{normal}
\phantomsection\label{\detokenize{index::doc}}
\begin{sphinxadmonition}{note}{Note:}
อัพเดทล่าสุด: 4 มีนาคม 2021
\end{sphinxadmonition}



คณะวิทยาศาสตร์และเทคโนโลยี มหาวิทยาลัยเทคโนโลยีราชมงคลพระนคร

เอกสารและเว็บไซต์นี้ จัดทำขึ้นเพื่อรวบรวมข้อมูลที่สำคัญทั้งหมดในการประเมินผลสัมฤทธิ์ของพนักงานมหาวิทยาลัย โดยประกอบเป็น 4 ส่วนหลักด้วยกัน ได้แก่
\begin{itemize}
\item {} 
ขั้นตอนการส่งหลักฐาน

\item {} 
การคำนวณภาระงาน

\item {} 
คำจำกัดความ

\item {} 
เอกสารประกอบทั้งหมด

\end{itemize}


\chapter{การประเมินผลสัมฤทธิ์ของงานของพนักงานมหาวิทยาลัย (องค์ประกอบที่ 1)  ตำแหน่งวิชาการ}
\label{\detokenize{submission_part1:id1}}\label{\detokenize{submission_part1::doc}}
คะแนนการประเมิน แบ่งได้ดังต่อไปนี้
\begin{enumerate}
\sphinxsetlistlabels{\arabic}{enumi}{enumii}{}{.}%
\item {} 
งานสอน ร้อยละ 50

\item {} 
งานวิจัยและงานวิชาการอื่น ร้อยละ 20

\item {} 
งานบริการวิชาการ ร้อยละ 15

\item {} 
งานทำนุบำรุงศิลปวัฒนธรรม ร้อยละ 5

\item {} 
งานอื่น ๆ ร้อยละ 10

\end{enumerate}


\bigskip\hrule\bigskip



\section{การส่งหลักฐาน}
\label{\detokenize{submission_part1:id2}}

\subsection{1. งานสอน}
\label{\detokenize{submission_part1:id3}}

\subsubsection{มคอ. 3/4/5/6}
\label{\detokenize{submission_part1:id4}}
ใช้หลักฐานจากระบบบริการการศึกษา เมนูภาระงาน มคอ. เมนูย่อยบันทึก มคอ.3 /มคอ.4  และ มคอ.5/มคอ.6 ร่วมกับ ข้อมูลของฝ่ายวิชาการ วิจัยและบริการวิชาการ

หากมีการรายงานคณาจารย์ดำเนินการส่ง มอค. 3, 4, 5, 6 ช้ากว่าที่มหาวิทยาลัยเทคโนโลยี
ราชมงคลพระนครกำหนดไว้ให้หัวหน้าสาขาวิชา/หัวหน้าหมวดวิชาดำเนินการพิจารณาประเมินพฤติกรรมการปฏิบัติราชการ (องค์ประกอบที่ 2)


\subsubsection{สื่อการสอน}
\label{\detokenize{submission_part1:id5}}\begin{itemize}
\item {} 
อ่านนิยามของสื่อการสอนได้ที่ {\hyperref[\detokenize{glossary:term-4}]{\sphinxtermref{\DUrole{xref,std,std-term}{สื่อการสอน}}}}

\item {} 
ส่งหลักฐานสื่อการสอน \sphinxstylestrong{ทุก} วิชา ตามตารางสอนในรอบการประเมิน

\item {} 
ระบุชื่อของอาจารย์ผู้สอน และสาขาวิชาหรือหมวดวิชาในสื่อการสอนให้ชัดเจน

\item {} 
รายละเอียดหลักฐานของสื่อการสอนแต่ละประเภท มีดังนี้

\end{itemize}
\begin{enumerate}
\sphinxsetlistlabels{\arabic}{enumi}{enumii}{}{.}%
\item {} 
{\hyperref[\detokenize{glossary:term-0}]{\sphinxtermref{\DUrole{xref,std,std-term}{เอกสารประกอบการสอน}}}} {\hyperref[\detokenize{glossary:term-1}]{\sphinxtermref{\DUrole{xref,std,std-term}{เอกสารคำสอน}}}} {\hyperref[\detokenize{glossary:term-2}]{\sphinxtermref{\DUrole{xref,std,std-term}{หนังสือ}}}} และ {\hyperref[\detokenize{glossary:term-3}]{\sphinxtermref{\DUrole{xref,std,std-term}{ตำรา}}}} ใช้เอกสารใดเอกสารหนึ่งต่อไปนี้
\begin{itemize}
\item {} 
{\hyperref[\detokenize{glossary:term-7}]{\sphinxtermref{\DUrole{xref,std,std-term}{แบบรับรองการเผยแพร่ผลงานทางวิชาการ}}}}  หรือ

\item {} 
{\hyperref[\detokenize{glossary:term-8}]{\sphinxtermref{\DUrole{xref,std,std-term}{แบบรับรองภาระงานทางวิชาการ}}}}  หรือ

\item {} 
หน้าปก คำนำ และสารบัญ

\end{itemize}

\item {} 
เอกสารนำเสนอ(พาวเวอร์พอยท์)
\begin{itemize}
\item {} 
หน้าปก และ เนื้อหาจำนวน ไม่ต่ำกว่า 1 หน่วยเรียน

\end{itemize}

\item {} 
บทเรียนในระบบอินเตอร์เน็ต
\begin{itemize}
\item {} 
ลิงก์ที่สามารเข้าถึงไฟล์อิเล็กทรอนิกส์ของเนื้อหาที่ระบุในบทเรียนในระบบอินเตอร์เน็ต

\end{itemize}

\item {} 
{\hyperref[\detokenize{glossary:term-2}]{\sphinxtermref{\DUrole{xref,std,std-term}{หนังสือ}}}} กรณีได้รับการจัดพิมพ์และจัดจำหน่ายโดยสำนักพิมพ์ที่ได้รับรองทางวิชาการ
\begin{itemize}
\item {} 
หลักฐานว่าได้ผ่านการประเมินโดยคณะผู้ทรงคุณวุฒิในสาขาวิชานั้น ๆ หรือสาขาวิชาที่เกี่ยวข้อง (peer reviewer) ที่มาจากหลากหลายสถาบัน และ

\item {} 
หน้าปกนอก และ หน้าแสดงลิขสิทธิ์และบรรณานุกรมของหนังสือ (CIP)

\end{itemize}

\end{enumerate}


\subsection{2. งานวิจัยและงานวิชาการอื่น}
\label{\detokenize{submission_part1:id6}}\begin{itemize}
\item {} 
ส่งผลงานหลัก 1 ผลงาน ที่มีคุณภาพในระดับสูงที่สุด พร้อมระบุระดับ

\item {} 
ส่งผลงานรองจำนวนเท่าไหร่ก็ได้ เพื่อพิจารณา

\item {} 
กรณีส่ง {\hyperref[\detokenize{glossary:term-0}]{\sphinxtermref{\DUrole{xref,std,std-term}{เอกสารประกอบการสอน}}}} หรือ {\hyperref[\detokenize{glossary:term-1}]{\sphinxtermref{\DUrole{xref,std,std-term}{เอกสารคำสอน}}}} \sphinxstylestrong{ต้อง} มีหนังสือบันทึกข้อความเสนอคณบดีเผยแพร่ก่อนเปิดภาคเรียน

\end{itemize}


\subsection{3. งานบริการวิชาการ}
\label{\detokenize{submission_part1:id7}}
ใช้หลักฐานที่เกี่ยวข้องกับงานบริการวิชาการที่แสดงถึงการมีส่วนร่วมในโครงการบริการวิชาการภายในและภายนอกหน่วยงาน ได้แก่
\begin{itemize}
\item {} 
คำสั่งปฏิบัติงาน

\item {} 
หนังสือเชิญเป็นวิทยากร

\end{itemize}

ในรูปแบบใดรูปแบบหนึ่งต่อไปนี้
\begin{itemize}
\item {} 
ภาพถ่าย

\item {} 
ไฟล์อิเล็กทรอนิกส์ (PDF)

\end{itemize}


\subsection{4. งานทำนุบำรุงศิลปวัฒนธรรม}
\label{\detokenize{submission_part1:id8}}
ใช้หลักฐานที่เกี่ยวข้องกับงานทำนุบำรุงศิลปวัฒนธรรม เช่น
\begin{itemize}
\item {} 
คำสั่งปฏิบัติงาน

\item {} 
หนังสือขออนุญาตเข้าร่วมงาน/กิจกรรม

\item {} 
ไฟล์อิเล็กทรอนิกส์ (PDF) ที่แสดงถึงงานทำนุบำรุงศิลปวัฒนธรรม

\end{itemize}

เป็นต้น


\subsection{5. งานอื่น ๆ}
\label{\detokenize{submission_part1:id9}}
ใช้หลักฐานที่เกี่ยวข้องกับงานอื่นๆหรืองานที่ได้รับมอบหมายหรือได้รับอนุญาตที่นอกเหนือจาก งานสอน งานวิจัยและวิชาการอื่น งานบริการวิชาการ และงานทำนุบำรุงศิลปวัฒนธรรม เช่น
\begin{itemize}
\item {} 
หนังสือขออนุญาตเข้าร่วมงาน/กิจกรรม

\item {} 
ไฟล์อิเล็กทรอนิกส์ (PDF) ที่แสดงถึงการเข้าร่วมงานอื่นๆหรืองานที่ได้รับมอบหมายหรือได้รับอนุญาต

\item {} 
ถ้าเป็นระดับ 5 ให้ส่งทั้งหมดที่มี (ส่งมากกว่า 5)

\end{itemize}


\bigskip\hrule\bigskip



\section{ระดับความสำเร็จในการจัดทำ}
\label{\detokenize{submission_part1:id10}}

\subsection{1. งานสอน}
\label{\detokenize{submission_part1:id11}}\begin{description}
\item[{ระดับที่ 1}] \leavevmode
มี มคอ.3 และ/หรือ มคอ.4 ประจำรายวิชาสอนที่เป็นไปตามข้อกำหนด/ตามแบบฟอร์มที่ มทร.พระนคร กำหนด หรือมีโครงการสอนในหลักสูตรที่ไม่ใช่หลักสูตร TQF

\item[{ระดับที่ 2}] \leavevmode
เป็นไปตามระดับที่ 1 และมีการพัฒนาสื่อการสอนประกอบโครงการสอน หรือ มีการเรียนการสอนตาม มคอ. 3 และ/หรือ มคอ.4

\item[{ระดับที่ 3}] \leavevmode
เป็นไปตามระดับที่ 2 และมีการสอบวัดผลการศึกษาตามระเบียบของ มหาวิทยาลัย และประกาศมหาวิทยาลัยเทคโนโลยีราชมงคลพระนคร เรื่องเกณฑ์การวัดและประเมินผล

\item[{ระดับที่ 4}] \leavevmode
เป็นไปตามระดับที่ 3  และมีการจัดการเรียนการสอนที่เน้นผู้เรียนเป็นสำคัญ อาทิเช่น การสอนแบบแก้ปัญหา รูปแบบการเรียนที่ใช้
ปัญหาเป็นหลัก วิธีสอนแบบระดมพลังสมอง วิธีสอนแบบบูรณาการ ฯลฯ

\item[{ระดับที่ 5}] \leavevmode
เป็นไปตามระดับที่ 4 และมีการจัดทำ มคอ.5 และ/หรือ มคอ.6 รวมทั้งมีการนำผลไปปรับปรุงการสอน

\end{description}


\subsection{2. งานวิจัยและงานวิชาการอื่น}
\label{\detokenize{submission_part1:id12}}
เลือกข้อใดข้อหนึ่งระหว่าง
\begin{itemize}
\item {} 
2.1 งานวิจัย

\item {} 
2.2 งานวิชาการอื่น

\end{itemize}


\subsubsection{ระดับความสำเร็จในการจัดทำงานวิจัย}
\label{\detokenize{submission_part1:id13}}\begin{description}
\item[{ระดับที่ 1}] \leavevmode
มีหัวข้องานวิจัยหรือข้อมูลเบื้องต้นที่จะนำไปสู่หัวข้องานวิจัย

\item[{ระดับที่ 2}] \leavevmode
เป็นไปตามระดับที่ 1 และเสนอโครงการวิจัยต่อหน่วยงาน

\item[{ระดับที่ 3}] \leavevmode
เป็นไปตามระดับที่ 2 และได้รับการอนุมัติให้ดำเนินการวิจัย

\item[{ระดับที่ 4}] \leavevmode
เป็นไปตามระดับที่ 3 และดำเนินการวิจัยเสร็จตามกำหนด

\item[{ระดับที่ 5}] \leavevmode
เป็นไปตามระดับที่ 4 และมีการเผยแพร่ผลงานวิจัย/การจัดนิทรรศการ/การนำเสนอผลงาน ในที่ประชุมระดับชาติ หรือนานาชาติ

\end{description}


\subsubsection{ระดับความสำเร็จในการจัดทำงานวิชาการอื่น}
\label{\detokenize{submission_part1:id14}}\begin{description}
\item[{ระดับที่ 1}] \leavevmode
มีชื่อ/หัวข้อ {\hyperref[\detokenize{glossary:term-3}]{\sphinxtermref{\DUrole{xref,std,std-term}{ตำรา}}}}/{\hyperref[\detokenize{glossary:term-2}]{\sphinxtermref{\DUrole{xref,std,std-term}{หนังสือ}}}}/บทความวิชาการ/บทความวิจัยหรือข้อมูลสำหรับการเขียนผลงานดังกล่าว

หรือ มีร่าง {\hyperref[\detokenize{glossary:term-0}]{\sphinxtermref{\DUrole{xref,std,std-term}{เอกสารประกอบการสอน}}}}

\item[{ระดับที่ 2}] \leavevmode
เป็นไปตามระดับที่ 1 และจัดทำเค้าโครงของ {\hyperref[\detokenize{glossary:term-3}]{\sphinxtermref{\DUrole{xref,std,std-term}{ตำรา}}}}/{\hyperref[\detokenize{glossary:term-2}]{\sphinxtermref{\DUrole{xref,std,std-term}{หนังสือ}}}}/บทความวิชาการ/บทความวิจัยที่ครบถ้วน

หรือ มี {\hyperref[\detokenize{glossary:term-0}]{\sphinxtermref{\DUrole{xref,std,std-term}{เอกสารประกอบการสอน}}}}  หรือ {\hyperref[\detokenize{glossary:term-1}]{\sphinxtermref{\DUrole{xref,std,std-term}{เอกสารคำสอน}}}} ฉบับสมบูรณ์พร้อมเผยแพร่

\item[{ระดับที่ 3}] \leavevmode
เป็นไปตามระดับที่ 2 และจัดทำ {\hyperref[\detokenize{glossary:term-3}]{\sphinxtermref{\DUrole{xref,std,std-term}{ตำรา}}}}/{\hyperref[\detokenize{glossary:term-2}]{\sphinxtermref{\DUrole{xref,std,std-term}{หนังสือ}}}}/บทความวิชาการ/บทความวิจัยฉบับร่าง

\item[{ระดับที่ 4}] \leavevmode
เป็นไปตามระดับที่ 3 และจัดทำ {\hyperref[\detokenize{glossary:term-3}]{\sphinxtermref{\DUrole{xref,std,std-term}{ตำรา}}}}/{\hyperref[\detokenize{glossary:term-2}]{\sphinxtermref{\DUrole{xref,std,std-term}{หนังสือ}}}}/บทความวิชาการ/บทความวิจัยฉบับสมบูรณ์พร้อมเผยแพร่

\item[{ระดับที่ 5}] \leavevmode
เป็นไปตามระดับที่ 4 และมีการเผยแพร่ผลงาน {\hyperref[\detokenize{glossary:term-3}]{\sphinxtermref{\DUrole{xref,std,std-term}{ตำรา}}}}/ {\hyperref[\detokenize{glossary:term-2}]{\sphinxtermref{\DUrole{xref,std,std-term}{หนังสือ}}}}/บทความวิชาการ/บทความวิจัยผ่านสื่อสิ่งพิมพ์หรือมิใช่สิ่งพิมพ์

\end{description}

\begin{sphinxadmonition}{note}{Note:}
ระดับที่ 5 ใช้หลักเกณฑ์การเผยแพร่ตามประกาศ ก.พ.อ เรื่อง หลักเกณฑ์และวิธีการพิจารณาแต่งตั้งบุคคลให้ดำรง ตำแหน่ง ผู้ช่วยศาสตราจารย์
รองศาสตราจารย์ ศาสตราจารย์ พ.ศ. 2561 หรือ พ.ศ. 2563 อย่างใดอย่างหนึ่ง
\end{sphinxadmonition}

\begin{sphinxadmonition}{note}{Note:}
ผลงานทางวิชาการเป็นไปตาม ประกาศ ก.พ.อ เรื่อง หลักเกณฑ์และวิธีการพิจารณาแต่งตั้งบุคคลให้ดำรงตำแหน่ง ผู้ช่วยศาสตราจารย์ รองศาสตราจารย์ ศาสตราจารย์ พ.ศ. 2561 หรือ พ.ศ. 2563
\end{sphinxadmonition}

\begin{sphinxadmonition}{note}{Note:}
{\hyperref[\detokenize{glossary:term-0}]{\sphinxtermref{\DUrole{xref,std,std-term}{เอกสารประกอบการสอน}}}} หรือ {\hyperref[\detokenize{glossary:term-1}]{\sphinxtermref{\DUrole{xref,std,std-term}{เอกสารคำสอน}}}} จะต้องมี {\hyperref[\detokenize{glossary:term-7}]{\sphinxtermref{\DUrole{xref,std,std-term}{แบบรับรองการเผยแพร่ผลงานทางวิชาการ}}}} ในภาคเรียนนั้นด้วย
\end{sphinxadmonition}


\subsection{3. ภาระงานบริการวิชาการ}
\label{\detokenize{submission_part1:id15}}\begin{description}
\item[{ระดับที่ 1}] \leavevmode
เป็นคณะกรรมการหรือผู้เข้าร่วมโครงการกับหน่วยงานภายใน 1 โครงการ

\item[{ระดับที่ 2}] \leavevmode
เป็นคณะกรรมการหรือผู้เข้าร่วมโครงการกับหน่วยงานภายในอย่างน้อย 2 โครงการ

\item[{ระดับที่ 3}] \leavevmode
เป็นคณะกรรมการหรือผู้เข้าร่วมโครงการกับหน่วยงานภายนอกอย่างน้อย 1 โครงการ

\sphinxstyleemphasis{หรือ} เป็น {\hyperref[\detokenize{glossary:term-5}]{\sphinxtermref{\DUrole{xref,std,std-term}{ผู้มีส่วนร่วมในโครงการ}}}} (ผู้รับผิดชอบโครงการ วิทยากร หรือ ผู้ช่วยวิทยากร) ภายในหน่วยงานหรือกับหน่วยงานภายนอก

\item[{ระดับที่ 4}] \leavevmode
{\hyperref[\detokenize{glossary:term-11}]{\sphinxtermref{\DUrole{xref,std,std-term}{การบูรณาการ}}}} งานบริการวิชาการร่วมกับการเรียนการสอน หรืองานวิชาการอื่น

\end{description}
\begin{description}
\item[{ระดับที่ 5}] \leavevmode
การบริการทางวิชาการตามระเบียบฯ (ดูไฟล์) ว่าด้วยการให้บริการสังคม

\end{description}


\bigskip\hrule\bigskip



\subsection{4. งานทำนุบำรุงศิลปวัฒนธรรม}
\label{\detokenize{submission_part1:id16}}\begin{description}
\item[{ระดับที่ 1}] \leavevmode
เป็นผู้มีส่วนร่วมงานทำนุบำรุงศิลปวัฒนธรรมในกิจกรรม/โครงการของหน่วยงาน หรือมหาวิทยาลัย จำนวน 1 กิจกรรม

\item[{ระดับที่ 2}] \leavevmode
เป็นผู้มีส่วนร่วมงานทำนุบำรุงศิลปวัฒนธรรมในกิจกรรม/โครงการของหน่วยงาน หรือมหาวิทยาลัย จำนวน 2 กิจกรรม

\item[{ระดับที่ 3}] \leavevmode
เป็นผู้มีส่วนร่วมงานทำนุบำรุงศิลปวัฒนธรรมในกิจกรรม/โครงการของหน่วยงาน หรือมหาวิทยาลัย จำนวน 3 กิจกรรม

\sphinxstyleemphasis{หรือ} เป็นคณะกรรมการงานทำนุบำรุงศิลปวัฒนธรรมในกิจกรรม/โครงการของหน่วยงาน หรือมหาวิทยาลัย จำนวน 1 โครงการ

\item[{ระดับที่ 4}] \leavevmode
เป็นผู้มีส่วนร่วมงานทำนุบำรุงศิลปวัฒนธรรมในกิจกรรม/โครงการของหน่วยงาน หรือมหาวิทยาลัย จำนวน 4 กิจกรรมขึ้นไป

\sphinxstyleemphasis{หรือ} เป็นคณะกรรมการงานทำนุบำรุงศิลปวัฒนธรรมในกิจกรรม/โครงการของหน่วยงาน หรือมหาวิทยาลัย จำนวน 2 โครงการขึ้นไป

\item[{ระดับที่ 5}] \leavevmode
{\hyperref[\detokenize{glossary:term-11}]{\sphinxtermref{\DUrole{xref,std,std-term}{การบูรณาการ}}}} ศิลปวัฒนธรรมกับหน่วยงานภายนอก

\end{description}


\subsection{5. งานอื่น ๆ}
\label{\detokenize{submission_part1:id17}}\begin{description}
\item[{ระดับที่ 1}] \leavevmode
เข้าร่วมกิจกรรม/งานอื่น ๆ หรืองานที่ได้รับมอบหมายหรือได้รับอนุญาตจากหน่วยงาน หรือมหาวิทยาลัย จำนวน 1 กิจกรรม

\item[{ระดับที่ 2}] \leavevmode
เข้าร่วมกิจกรรม/งานอื่น ๆ หรืองานที่ได้รับมอบหมายหรือได้รับอนุญาตจากหน่วยงาน หรือมหาวิทยาลัย จำนวน 2 กิจกรรม

\item[{ระดับที่ 3}] \leavevmode
เข้าร่วมกิจกรรม/งานอื่น ๆ หรืองานที่ได้รับมอบหมายหรือได้รับอนุญาตจากหน่วยงาน หรือมหาวิทยาลัย จำนวน 3 กิจกรรม

\item[{ระดับที่ 4}] \leavevmode
เข้าร่วมกิจกรรม/งานอื่น ๆ หรืองานที่ได้รับมอบหมายหรือได้รับอนุญาตจากหน่วยงาน หรือมหาวิทยาลัย จำนวน 4 กิจกรรม

\item[{ระดับที่ 5}] \leavevmode
เข้าร่วมกิจกรรม/งานอื่น ๆ หรืองานที่ได้รับมอบหมายหรือได้รับอนุญาตจากหน่วยงาน หรือมหาวิทยาลัย จำนวน 5 กิจกรรมขึ้นไป

\end{description}


\chapter{การนำส่งหลักฐานการประเมิน (สายวิชาการ) องค์ 2}
\label{\detokenize{submission_part2:id1}}\label{\detokenize{submission_part2::doc}}

\chapter{1. ภาระงานสอน}
\label{\detokenize{1teaching:id1}}\label{\detokenize{1teaching::doc}}\begin{description}
\item[{จำนวนชั่วโมงรวม}] \leavevmode
อย่างน้อย 9 ชั่วโมงต่อสัปดาห์ ไม่เกิน 25 ชั่วโมงต่อสัปดาห์

\end{description}
\begin{itemize}
\item {} 
ถ้าอาจารย์สอนหลายวิชา แต่ลงตารางสอนในเวลาเดียวกัน จะนับเพียงวิชาเดียวเท่านั้น ไม่นับซ้อนกัน
\begin{quote}

เช่น ถ้าในตารางสอนระบุดังนี้
วิชา ก. วันจันทร์ 9\sphinxhyphen{}12
วิชา ข. วันจันทร์ 9\sphinxhyphen{}12
ให้เลือกนับวิชาเดียวเท่านั้น
\end{quote}

\end{itemize}


\bigskip\hrule\bigskip



\section{1.1 งานสอนนักศึกษาในระดับปริญญาตรี}
\label{\detokenize{1teaching:id2}}

\begin{savenotes}\sphinxattablestart
\centering
\begin{tabular}[t]{|\X{20}{120}|\X{20}{120}|\X{20}{120}|\X{60}{120}|}
\hline
\sphinxstyletheadfamily 
ประเภทห้องเรียน
&\sphinxstyletheadfamily 
ภาคเรียนปกติ
&\sphinxstyletheadfamily 
ภาคเรียนฤดูร้อน
&\sphinxstyletheadfamily 
หมายเหตุ
\\
\hline
ภาคปกติ
&
1.1.1 ก
&&\\
\hline
ภาคสมทบ
&
1.1.1 ข
&&\\
\hline
ปรับวุฒิ
&
1.1.1 ข
&&\\
\hline
\end{tabular}
\par
\sphinxattableend\end{savenotes}

ในกรณีที่รายวิชาใดมีผู้สอนที่มากกว่าหนึ่งคนให้คำนวณภาระงานตามสัดส่วนจำนวนคาบ หรือจำนวนชั่วโมงที่คณาจารย์ประจำผู้นั้นได้ทำการสอนจริงต่อจำนวนคาบหรือจำนวนชั่วโมงโดยรวมทั้งหมดของรายวิชานั้น


\subsection{1.1.1 การสอนรายวิชาบรรยาย 1 หน่วยกิต}
\label{\detokenize{1teaching:id3}}\begin{description}
\item[{(ก) ภาคปกติ}] \leavevmode
3 ชั่วโมงต่อสัปดาห์

\end{description}

ครอบคลุมถึงภาระงานบรรยาย 1 ชั่วโมง การเตรียมการสอน 1 ชั่วโมง และการตรวจงานนักศึกษา 1 ชั่วโมง
\begin{description}
\item[{(ข) ภาคพิเศษและภาคนอกเวลา}] \leavevmode
1 ชั่วโมงต่อสัปดาห์

\end{description}


\subsection{1.1.2 การสอนรายวิชาปฏิบัติการ 1 หน่วยกิต}
\label{\detokenize{1teaching:id4}}\begin{description}
\item[{(ก) ภาคปกติ}] \leavevmode
3.5 ชั่วโมงต่อสัปดาห์

\end{description}

ครอบคลุมถึงภาระงานบรรยาย 1 ชั่วโมง การเตรียมการสอน 1 ชั่วโมง และการตรวจงานนักศึกษา 1 ชั่วโมง
\begin{description}
\item[{(ข) ภาคพิเศษและภาคนอกเวลา}] \leavevmode
1 ชั่วโมงต่อสัปดาห์

\end{description}


\subsection{1.1.3 งานสหกิจศึกษา/ฝึกงาน 1 หน่วยกิตอาจารย์นิเทศสหกิจ/ฝึกงาน}
\label{\detokenize{1teaching:id5}}\begin{quote}

1 ชั่วโมงต่อสัปดาห์
\end{quote}


\subsection{1.1.4 การศึกษาดูงาน}
\label{\detokenize{1teaching:id6}}\begin{description}
\item[{(ก) อาจารย์ผู้รับผิดชอบ}] \leavevmode
0.5 ชั่วโมงต่อสัปดาห์

\end{description}

ให้คำนวณตามจำนวนกิจกรรมหรือโครงการ
\begin{description}
\item[{(ข) อาจารย์ผู้ร่วมกิจกรรม}] \leavevmode
0.25 ชั่วโมงต่อสัปดาห์

\end{description}


\subsection{1.1.5 งานฝึกภาคสนาม}
\label{\detokenize{1teaching:id7}}\begin{description}
\item[{อาจารย์ผู้ควบคุม}] \leavevmode
1 ชั่วโมงต่อสัปดาห์

\end{description}

ให้คำนวณตามจำนวนกิจกรรมหรือโครงการ


\section{1.2 งานสอนนักศึกษาในระดับบัณฑิตศึกษา}
\label{\detokenize{1teaching:id8}}

\subsection{1.2.1 การสอนรายวิชาบรรยาย 1 หน่วยกิต}
\label{\detokenize{1teaching:id9}}\begin{description}
\item[{(ก) ภาคปกติ}] \leavevmode
4 ชั่วโมงต่อสัปดาห์

\item[{(ข) ภาคพิเศษและภาคนอกเวลา}] \leavevmode
1 ชั่วโมงต่อสัปดาห์

\end{description}

ครอบคลุมถึงภาระงานบรรยาย 2 ชม. การเตรียมการสอน 1 ชม. และการตรวจงานนักศึกษา 1 ชม.


\subsection{1.2.2 การสอนรายวิชาปฏิบัติ 1 หน่วยกิต}
\label{\detokenize{1teaching:id10}}\begin{description}
\item[{(ก) ภาคปกติ}] \leavevmode
5 ชั่วโมงต่อสัปดาห์

\item[{(ข) ภาคพิเศษและภาคนอกเวลา}] \leavevmode
2 ชั่วโมงต่อสัปดาห์

\end{description}

ครอบคลุมถึงภาระงานบรรยาย 2 ชม.
การเตรียมการสอน 1.5 ชม. และการตรวจงานนักศึกษา 1.5 ชม.


\section{1.3 งานสอนในระดับต่ำกว่าปริญญาตรี}
\label{\detokenize{1teaching:id11}}

\subsection{1.3.1 การสอนรายวิชาบรรยาย 1 หน่วยกิต}
\label{\detokenize{1teaching:id12}}\begin{quote}

1 ชั่วโมงต่อสัปดาห์
\end{quote}


\subsection{1.3.2 การสอนรายวิชาปฏิบัติการ 1 หน่วยกิต}
\label{\detokenize{1teaching:id13}}\begin{quote}

1 ชั่วโมงต่อสัปดาห์
\end{quote}


\section{1.4 งานด้านโครงงาน}
\label{\detokenize{1teaching:id14}}
ให้คำนวณตามจำนวนเรื่อง


\subsection{1.4.1 ที่ปรึกษาระดับปริญญาตรี}
\label{\detokenize{1teaching:id15}}\begin{description}
\item[{(ก) ที่ปรึกษาหลักระดับปริญญาตรี}] \leavevmode
2 ชั่วโมงต่อสัปดาห์

\item[{(ข) ที่ปรึกษาร่วมระดับปริญญาตรี}] \leavevmode
0.5 ชั่วโมงต่อสัปดาห์

\end{description}

อาจารย์ที่ปรึกษาหลักโครงงานฯ คือคณะกรรมการที่อยู่ในคำสั่ง ส่วนอาจารย์ที่ปรึกษาร่วมโครงงานฯ อยู่ในคำสั่งหรือบันทึกข้อความ


\subsection{1.4.2 กรรมการสอบโครงงาน}
\label{\detokenize{1teaching:id16}}\begin{description}
\item[{กรรมการสอบโครงงาน}] \leavevmode
0.5 ชั่วโมงต่อสัปดาห์

\end{description}


\section{1.5 งานด้านวิทยานิพนธ์และการค้นคว้าอิสระ}
\label{\detokenize{1teaching:id17}}

\subsection{1.5.1 งานที่ปรึกษา งานที่ปรึกษาเอกัตศึกษาหรืองานที่ปรึกษาอื่นที่เทียบเท่า ตามที่กำหนดไว้ในหลักสูตรระดับบัณฑิตศึกษาของมหาวิทยาลัย}
\label{\detokenize{1teaching:id18}}\begin{description}
\item[{(ก) ที่ปรึกษาระดับปริญญาโท}] \leavevmode
4 ชั่วโมงต่อสัปดาห์

\item[{(ข) ที่ปรึกษาระดับปริญญาเอก}] \leavevmode
5 ชั่วโมงต่อสัปดาห์

\item[{(ค) ที่ปรึกษาการค้นคว้าอิสระ}] \leavevmode
3 ชั่วโมงต่อสัปดาห์

\end{description}
\begin{itemize}
\item {} 
ให้คำนวณตามจำนวนนักศึกษา

\item {} 
ภาคพิเศษและภาคนอกเวลาให้คิดภาระงานครึ่งหนึ่งของภาคปกติ

\end{itemize}


\subsection{1.5.2 กรรมการสอบวิทยานิพนธ์}
\label{\detokenize{1teaching:id19}}\begin{description}
\item[{(ก) กรรมการสอบวิทยานิพนธ์ภาคปกติ}] \leavevmode
3 ชั่วโมงต่อสัปดาห์

\item[{(ข) กรรมการสอบการค้นคว้าอิสระภาคปกติ}] \leavevmode
0.5 ชั่วโมงต่อสัปดาห์

\end{description}

ภาคพิเศษและภาคนอกเวลาให้คิดภาระงานครึ่งหนึ่งของภาคปกติ


\section{1.6 การสอนในหลักสูตรอื่นนอกจาก 1.1 \sphinxhyphen{} 1.5}
\label{\detokenize{1teaching:id20}}
1 หน่วยกิตการจัดการสอนภาคทฤษฎีเท่ากับ 1 หน่วยกิตของรายวิชาบรรยาย (ภาคปกติ)
\begin{description}
\item[{(ก) การสอนภาคทฤษฎี}] \leavevmode
จำนวนหน่วยกิต = จำนวนชั่วโมงของหลักสูตร หาร 15

ภาระงานภาคทฤษฎี = จำนวนหน่วยกิต คูณ 3

\item[{(ข) การสอนภาคปฏิบัติ}] \leavevmode
จำนวนหน่วยกิต = จำนวนชั่วโมงของหลักสูตร หาร 30

ภาระงานภาคทฤษฎี = จำนวนหน่วยกิต คูณ 3.5

\item[{(ค) การสอนฝึกปฏิบัติในโรงฝึกงานหรือภาคสนาม}] \leavevmode
จำนวนหน่วยกิต = จำนวนชั่วโมงของหลักสูตร หาร 45

ภาระงานภาคทฤษฎี = จำนวนหน่วยกิต คูณ 3.5

\end{description}


\chapter{2. ภาระงานวิจัยและบริการวิชาการอื่น}
\label{\detokenize{2research:id1}}\label{\detokenize{2research::doc}}\begin{description}
\item[{จำนวนชั่วโมงรวม}] \leavevmode
ไม่เกิน 10 ชั่วโมงต่อสัปดาห์

\end{description}


\section{เงื่อนไขการได้รับการประเมิน}
\label{\detokenize{2research:id2}}\begin{enumerate}
\sphinxsetlistlabels{\arabic}{enumi}{enumii}{}{.}%
\item {} 
ร่างงานวิจัย หรือ ร่างบทความวิชาการ \sphinxstylestrong{ไม่} สามารถนำมาคิดภาระงานได้

\item {} 
กรณีมีผู้ร่วมโครงการมากกว่า 1 คน ให้คิดภาระงานตามร้อยละของการมีส่วนร่วมในผลงานนั้นๆ โดยต้องนำส่งเอกสารแสดงการแบ่งร้อยละของการมีส่วนร่วมที่มีการลงนามของผู้ที่มีรายชื่อ ทุกคนในผลงานนั้นๆ ให้แก่หน่วยงานพิจารณากรณีตาม 2.1 และ 2.2 ให้คิดภาระงานตลอดทั้งปี ส่วน 2.3 2.4 และ 2.5 ให้คิดภาระงานเฉพาะ ภาคการศึกษาที่ผลงานปรากฏ

\end{enumerate}


\section{สิ่งจำเป็นในหลักฐานการประเมิน}
\label{\detokenize{2research:id3}}\begin{enumerate}
\sphinxsetlistlabels{\arabic}{enumi}{enumii}{}{.}%
\item {} 
\sphinxstylestrong{ต้อง} คำนวณและใส่คะแนนภาระงาน ตามประกาศทางคณะวิทยาศาสตร์และเทคโนโลยี คณะวิทยาศาสตร์และเทคโนโลยีขอสงวนสิทธิ์ในการดำเนินการ

\item {} 
เนื่องจากผลงานใน {\hyperref[\detokenize{2research:id10}]{\sphinxcrossref{\DUrole{std,std-ref}{2.3}}}} สามารถเคลมได้ 2 รอบ ฉะนั้นต้องระบุเพิ่มเติมด้วยว่าเป็น \sphinxstyleemphasis{ครั้งที่ 1} หรือ \sphinxstyleemphasis{ครั้งที่ 2}

\item {} 
ระบุวันเวลาอย่างชัดเจน

\end{enumerate}


\bigskip\hrule\bigskip



\section{2.1 การร่วมทำวิจัย}
\label{\detokenize{2research:id4}}
ไม่นับโครงการที่มีการขยายระยะเวลา


\subsection{2.1.1 มีส่วนร่วมในโครงการวิจัยตั้งแต่ร้อยละ 60 ขึ้นไป}
\label{\detokenize{2research:id5}}
3.5 ชั่วโมงต่อสัปดาห์


\subsection{2.1.2 มีส่วนร่วมในโครงการวิจัยตั้งแต่ร้อยละ 40 \sphinxhyphen{} 59}
\label{\detokenize{2research:id6}}
2.5 ชั่วโมงต่อสัปดาห์


\subsection{2.1.3 มีส่วนร่วมในโครงการวิจัยตั้งแต่ร้อยละ 20 \sphinxhyphen{} 39}
\label{\detokenize{2research:id7}}
2 ชั่วโมงต่อสัปดาห์


\subsection{2.1.4 มีส่วนร่วมในโครงการวิจัยต่ำกว่าร้อยละ 20}
\label{\detokenize{2research:id8}}
1.5 ชั่วโมงต่อสัปดาห์


\section{2.2 ผู้อำนวยแผนวิจัย}
\label{\detokenize{2research:id9}}
2 ชั่วโมงต่อสัปดาห์


\section{2.3 การตีพิมพ์เผยแพร่บทความวิจัย}
\label{\detokenize{2research:id10}}\label{\detokenize{2research:id11}}
\begin{sphinxadmonition}{important}{Important:}
ผลงานในส่วนนี้ สามารถเคลมได้ 2 รอบการประเมินติดต่อกัน
\end{sphinxadmonition}


\subsection{2.3.1 การเข้าร่วมประชุมหรือสัมมนาทางวิชาการ ที่ผ่านการอนุมัติหรือเห็นชอบจากหน่วยงาน}
\label{\detokenize{2research:id12}}
0.5 ชั่วโมงต่อสัปดาห์

นับตามจำนวนผลงาน เช่น ถ้าร่วม 1 งานแต่นำเสนอ 3 ผลงาน ให้นับ 3 ผลงานแยกกัน

\begin{sphinxadmonition}{warning}{Warning:}
ต้องมีการนำเสนอผลงาน หรือถ้าเป็นอาจารย์ที่ปรึกษาโครงการวิจัยให้นักศึกษาที่ไปนำเสนอในงานประชุมวิชาการ จะต้องมีชื่ออยู่ในการนำเสนอนั้นด้วย
\end{sphinxadmonition}


\subsection{2.3.2 บทความวิจัยหรือบทความวิชาการฉบับสมบูรณ์ที่ตีพิมพ์ในรายงานสืบเนื่องจากการประชุมวิชาการระดับชาติ}
\label{\detokenize{2research:id13}}
0.75 ชั่วโมงต่อสัปดาห์


\subsection{2.3.3 บทความวิจัยหรือบทความวิชาการฉบับสมบูรณ์}
\label{\detokenize{2research:id14}}
ที่ตีพิมพ์ในรายงานสืบเนื่องจากการประชุมวิชาการระดับนานาชาติ หรือในวารสารทางวิชาการระดับชาติที่ไม่อยู่ในฐานข้อมูล ตามประกาศ ก.พ.อ. หรือระเบียบ คณะกรรมการการอุดมศึกษาว่าด้วยหลักเกณฑ์การพิจารณาวารสารทางวิชาการสำหรับการเผยแพร่ผลงานทางวิชาการ พ.ศ.๒๕๕๖ แต่สถาบันนำเสนอสภาสถาบันอนุมัติและจัดทำเป็นประกาศให้ทราบ เป็นการทั่วไป และแจ้งให้ กพอ./กกอ. ทราบภายใน ๓๐ วันนับแต่วันที่ออกประกาศ


\subsection{2.3.4 บทความวิจัยหรือบทความวิชาการที่ตีพิมพ์ในวารสารวิชาการที่ปรากฏในฐานข้อมูล TCI กลุ่มที่ 2}
\label{\detokenize{2research:tci-2}}
2 ชั่วโมงต่อสัปดาห์


\subsection{2.3.5 บทความวิจัยหรือบทความวิชาการที่ตีพิมพ์ ในวารสารวิชาการระดับนานาชาติที่ไม่อยู่ในฐานข้อมูล}
\label{\detokenize{2research:id15}}
เทียบประกาศ ก.พ.อ. หรือระเบียบ คณะกรรมการการอุดมศึกษาว่าด้วยหลักเกณฑ์ การพิจารณาวารสารทางวิชาการ สำหรับการเผยแพร่ผลงานทางวิชาการ พ.ศ. 2562 แต่สถาบันนำเสนอสภาสถาบันอนุมัติและจัดทำเป็นประกาศ ให้ทราบเป็นการทั่วไป และแจ้งให้ กพอ./กกอ. ทราบภายใน 30 วันนับแต่วันที่ออกประกาศ หรือตีพิมพ์ในวารสารวิชาการที่ปรากฏใน ฐานข้อมูล TCI กลุ่มที่ 1

2.5 ชั่วโมงต่อสัปดาห์


\subsection{2.3.6 บทความวิจัยหรือบทความวิชาการที่ตีพิมพ์ในวารสารวิชาการระดับนานาชาติที่ปรากฏในฐานข้อมูล}
\label{\detokenize{2research:id16}}
ระดับนานาชาติตามประกาศ ก.พ.อ. หรือระเบียบคณะกรรมการ การอุดมศึกษาว่าด้วยหลักเกณฑ์การพิจารณาวารสารทางวิชาการ สำหรับการเผยแพร่ผลงานทางวิชาการ พ.ศ. 2562

3 ชั่วโมงต่อสัปดาห์


\subsection{2.3.7 ผลงานที่ได้รับการจดอนุสิทธิบัตร}
\label{\detokenize{2research:id17}}
1.5 ชั่วโมงต่อสัปดาห์


\subsection{2.3.8 ผลงานได้รับการจดสิทธิบัตร}
\label{\detokenize{2research:id18}}
3 ชั่วโมงต่อสัปดาห์


\subsection{2.3.9 ผลงานวิชาการรับใช้สังคมที่ได้รับการประเมินผ่านเกณฑ์การขอตำแหน่งทางวิชาการแล้ว}
\label{\detokenize{2research:id19}}
3 ชั่วโมงต่อสัปดาห์


\subsection{2.3.10 ผลงานวิจัยที่หน่วยงานหรือองค์กรระดับชาติว่าจ้างให้ดำเนินการ}
\label{\detokenize{2research:id20}}
3 ชั่วโมงต่อสัปดาห์


\subsection{2.3.11 งานวิจัยที่ได้รับทุนสนับสนุนจากหน่วยงานภายนอก}
\label{\detokenize{2research:id21}}\begin{description}
\item[{(ก) หัวหน้าโครงการวิจัย}] \leavevmode
5 ชั่วโมงต่อสัปดาห์

\item[{(ข) มีส่วนร่วมในโครงการวิจัยตั้งแต่ร้อยละ 60 ขึ้นไป}] \leavevmode
4.5 ชั่วโมงต่อสัปดาห์

\item[{(ค) มีส่วนร่วมในโครงการวิจัยตั้งแต่ร้อยละ 40\sphinxhyphen{}59}] \leavevmode
3.5 ชั่วโมงต่อสัปดาห์

\item[{(ง) มีส่วนร่วมในโครงการวิจัยตั้งแต่ร้อยละ 20\sphinxhyphen{}39}] \leavevmode
3 ชั่วโมงต่อสัปดาห์

\item[{(จ) มีส่วนร่วมในโครงการวิจัยต่ำกว่าร้อยละ 20}] \leavevmode
2.5 ชั่วโมงต่อสัปดาห์

\end{description}

ให้คำนวณตามจำนวนโครงการโดยไม่นับโครงการที่มีการขยายระยะเวลา


\subsection{2.3.12 ผลงานค้นพบพันธุ์พืช พันธุ์สัตว์ และ/หรือ สิ่งมีชีวิต และ/หรือ สารประกอบทางธรรมชาติชนิดใหม่ และ/หรือ ปฏิกิริยาเคมี ที่ค้นพบใหม่และได้รับการจดทะเบียน}
\label{\detokenize{2research:id22}}
3 ชั่วโมงต่อสัปดาห์


\subsection{2.3.13 ตำราหรือหนังสือที่ได้รับการประเมินผ่านเกณฑ์การขอตำแหน่งทางวิชาการแล้ว}
\label{\detokenize{2research:id23}}
3 ชั่วโมงต่อสัปดาห์


\subsection{2.3.14 ตำราหรือหนังสือที่ผ่านการพิจารณาตามหลักเกณฑ์การประเมินตำแหน่งทางวิชาการแต่ไม่ได้นำมาขอรับการประเมินตำแหน่งทางวิชาการ}
\label{\detokenize{2research:id24}}
3 ชั่วโมงต่อสัปดาห์


\subsection{2.3.15 เอกสารประกอบการสอนหรือเอกสารคำสอน}
\label{\detokenize{2research:id25}}
2 ชั่วโมงต่อสัปดาห์

\begin{sphinxadmonition}{important}{Important:}
ถ้าจะส่งเอกสารประกอบการสอนใน 2 รอบการประเมิน จำเป็นจะต้องทำบันทึกข้อความที่แสดงว่า ระบุว่าในรอบที่ 2 นั้นมีความเปลี่ยนแปลงจากเดิมส่วนใดบ้าง
\end{sphinxadmonition}


\section{2.4 การเผยแพร่ผลงานสร้างสรรค์}
\label{\detokenize{2research:id26}}

\subsection{2.4.1 งานสร้างสรรค์ที่มีการเผยแพร่สู่สาธารณะในลักษณะใดลักษณะหนึ่งหรือผ่านสื่ออิเล็กทรอนิกส์ online}
\label{\detokenize{2research:online}}
0.5 ชั่วโมงต่อสัปดาห์


\subsection{2.4.2 งานสร้างสรรค์ที่ได้รับการเผยแพร่ในระดับสถาบัน}
\label{\detokenize{2research:id27}}
1 ชั่วโมงต่อสัปดาห์


\subsection{2.4.3 งานสร้างสรรค์ที่ได้รับการเผยแพร่ในระดับชาติ}
\label{\detokenize{2research:id28}}
1.5 ชั่วโมงต่อสัปดาห์


\subsection{2.4.4 งานสร้างสรรค์ที่ได้รับการเผยแพร่ในระดับความร่วมมือในระดับนานาชาติ}
\label{\detokenize{2research:id29}}
2 ชั่วโมงต่อสัปดาห์


\section{2.5 การนำผลงานวิจัยไปใช้ประโยชน์}
\label{\detokenize{2research:id30}}
2 ชั่วโมงต่อสัปดาห์

โดยให้คิดตามจำนวนผลงาน

การนำผลงานวิจัยไปใช้ประโยชน์นอกมหาวิทยาลัย ในส่วนราชการระดับกรม หรือเทียบเท่าขึ้นไป หรือรัฐวิสาหกิจ หรือองค์กรมหาชน หรือองค์กรระดับชาติ ทั้งภาครัฐและเอกชนในลักษณะเป็นการใช้ประโยชน์ ดังนี้ การใช้ประโยชน์เชิง สาธารณะการใช้ประโยชน์ เชิงนโยบายการใช้ประโยชน์เชิงพาณิชย์ และ การใช้ประโยชน์ทางอ้อมของงานสร้างสรรค์


\chapter{3. ภาระงานบริการวิชาการ}
\label{\detokenize{3service:id1}}\label{\detokenize{3service::doc}}\begin{description}
\item[{จำนวนชั่วโมงรวม}] \leavevmode
ไม่เกิน 5 ชั่วโมงต่อสัปดาห์

\end{description}


\section{การพิจารณาสัดส่วน}
\label{\detokenize{3service:id2}}
การคิดภาระงานบริการวิชาการ ให้พิจารณาตามสัดส่วนของการมีส่วนร่วมในงานบริการทางวิชาการนั้น ๆ โดยต้องมีเอกสารยืนยันการมีสัดส่วนผลงานจากผู้มีส่วนร่วมทุกคน

ผู้รับผิดชอบในโครงการบริการวิชาการคิดคำนวณคะแนนภาระงานให้ผู้ปฏิบัติงานในโครงการทุกคน โดยให้ผู้ที่รับผิดชอบงานแบบเดียวกันได้รับคะแนนภาระงานที่เท่ากัน

คิดภาระงานด้วยสูตรดังต่อไปนี้
\begin{quote}

(ชั่วโมงที่ปฏิบัติงานจริง) x (อัตราส่วนของหน้าที่) ÷ 15
\end{quote}

โดยที่
\begin{itemize}
\item {} 
ชั่วโมงที่ปฏิบัติงานจริง ไม่นับเวลาพักกลางวัน 1 ชั่วโมง และเวลาเตรียมงานที่อยู่นอกเหนือกำหนดการในตาราง

\item {} 
อัตราส่วนของหน้าที่ ระบุไว้ด้านล่าง

\item {} 
การหารด้วย 15 แสดงถึงการเฉลี่ยภาระงานนั้นใน 1 ภาคเรียน (15 สัปดาห์)

\end{itemize}

เมื่อคำนวณเสร็จแล้ว ให้ปัดเศษให้เป็นทศนิยม 2 ตำแหน่ง และถ้ามีมากกว่า 1 หน้าที่ในโครงการ ให้คิดภาระงานแยกตามหน้าที่

ตัวอย่างเช่น
\begin{itemize}
\item {} 
ผู้รับผิดชอบโครงการ เวลา 8:30 \sphinxhyphen{} 16:30 นับเป็น 8 ชั่วโมง หักเวลาพักกลางวัน 1 ชั่วโมง เหลือ 7 ชั่วโมง ฉะนั้นจะได้
\begin{quote}

7 x 0.7 ÷ 15 = 3.27
\end{quote}

\item {} 
ผู้ร่วมโครงการ เวลา 8:30 \sphinxhyphen{} 12:30 จะได้
\begin{quote}

4 x 0.3 ÷ 15 = 0.8
\end{quote}

\end{itemize}


\section{เงื่อนไขการได้รับการประเมิน}
\label{\detokenize{3service:id3}}\begin{enumerate}
\sphinxsetlistlabels{\arabic}{enumi}{enumii}{}{.}%
\item {} 
กรณีเป็น 3.2 การบริการวิชาการเชิงพาณิชย์ ให้แนบหลักฐานการนำส่งเงิน หรือ ใบเสร็จจากการเงินของคณะวิทยาศาสตร์หรือมหาวิทยาลัยเทคโนโลยีราชมงคลพระนคร อย่างชัดเจน

\item {} 
กรณีระยะเวลาโครงการบริการสังคมมีการคาบเกี่ยวรอบการประเมิน 2 รอบ ให้นับภาระงานเต็มได้ทั้งได้ 2 ทั้งรอบ
\begin{quote}

เช่น โครงการ ก. จัด 1 มีนาคม ถ
\end{quote}

\end{enumerate}


\section{สิ่งจำเป็นในหลักฐานการประเมิน}
\label{\detokenize{3service:id4}}\begin{itemize}
\item {} 
\sphinxstylestrong{ต้อง} ใส่เวลาและคำนวณคะแนนมาอย่างชัดเจน ถ้าขาดตกบกพร่อง คณะวิทยาศาสตร์และเทคโนโลยีขอสงวนสิทธิ์ในการดำเนินการ

\item {} 
การได้รับเชิญเป็นวิทยากรให้ส่งหลักฐานการไปปฏิบัติหน้าที่อย่างชัดเจน เช่น ภาพถ่าย

\item {} 
กรณีคณาจารย์ปฏิบัติหน้าที่หลายหน้าที่ในโครงการบริการวิชาการ ให้แยกงานออกจากกัน เช่น
\begin{quote}

โครงการ A (กรรมการดำเนินโครงการ)
โครงการ A (วิทยากร)
\end{quote}

\end{itemize}

ผู้มีส่วนร่วมในโครงการภายใน
\begin{itemize}
\item {} 
ใบคำสั่งแต่งตั้งคณะกรรมการ หรือ

\item {} 
หนังสือเชิญเป็นวิทยากร

\end{itemize}

ผู้เข้าร่วมโครงการภายใน
\begin{itemize}
\item {} 
ใบลงทะเบียนเข้าร่วมโครงการที่ระบุเวลาและมีลายเซ็น

\end{itemize}

ผู้มีส่วนร่วมในโครงการภายนอก
\begin{itemize}
\item {} 
หนังสือที่ได้รับอนุมัติจากหัวหน้าหน่วยงาน \sphinxstylestrong{และ}

\item {} 
ตารางกำหนดการบอกเวลาปฏิบัติงาน

\item {} 
ภาพถ่าย ในกรณีที่เป็นวิทยากร

\end{itemize}

ผู้เข้าร่วมภายนอก
\begin{itemize}
\item {} 
หนังสือที่ได้รับอนุมัติจากหัวหน้าหน่วยงาน \sphinxstylestrong{และ}

\item {} 
ใบลงทะเบียนเข้าร่วมโครงการที่ระบุเวลาและมีลายเซ็น

\item {} 
ภาพถ่าย

\end{itemize}


\bigskip\hrule\bigskip



\section{3.1 การบริการวิชาการแก่สังคม}
\label{\detokenize{3service:id5}}

\subsection{3.1.1 ปฏิบัติโครงการบริการวิชาการแก่สังคมตามแผน}
\label{\detokenize{3service:id6}}

\subsubsection{3.1.1.1 ผู้ร่วมกิจกรรมในโครงการ}
\label{\detokenize{3service:id7}}\begin{description}
\item[{(ก) ผู้รับผิดชอบโครงการ}] \leavevmode
ร้อยละ 70

\item[{(ข) กรรมการหรือผู้ร่วมโครงการ}] \leavevmode
ร้อยละ 30

\end{description}


\subsubsection{3.1.1.2 วิทยากร}
\label{\detokenize{3service:id8}}\begin{description}
\item[{(ก) วิทยากร}] \leavevmode
ร้อยละ 70

\item[{(ข) ผู้ช่วยวิทยากร}] \leavevmode
ร้อยละ 70

\end{description}


\subsection{3.1.2 มีส่วนร่วมในการบริการวิชาการแก่สังคมในระดับสถาบัน}
\label{\detokenize{3service:id9}}
1 ชั่วโมงต่อสัปดาห์


\subsection{3.1.3 การเป็นกรรมการเพื่อบริการวิชาการภายในหน่วยงาน}
\label{\detokenize{3service:id10}}
1 ชั่วโมงต่อสัปดาห์

การเป็นกรรมการเพื่อบริการวิชาการภายในหน่วยงาน หมายถึง การเป็นผู้พิจาณาผลงานทางวิชาการ โดยคิดต่อตามโครงการ/กิจกรรม


\subsection{3.1.4 การเป็นกรรมการภายนอก (บริการสาธารณะ)}
\label{\detokenize{3service:id11}}
1 ชั่วโมงต่อสัปดาห์

การเป็นกรรมการภายนอก (บริการสาธารณะ) ให้หมายรวมถึง การเป็นกรรมการสมาคมวิชาการหรือวิชาชีพการเป็นกรรมการสอบวิทยานิพนธ์ การเป็น
ผู้พิจารณาบทความทางวิชาการ การเป็นผู้พิจาณาผลงานทางวิชาการ การเป็นกรรมการในบริการวิชาการเชิงพาณิชย์ ในข้อ 3.2.1 3.2.2 3.2.3 3.2.5
โดยให้คิดตามกิจกรรม/โครงการ/จำนวนบทความวิชาการ


\bigskip\hrule\bigskip



\section{3.2 การบริการวิชาการเชิงพาณิชย์ ที่มีการเซ็นสัญญาที่หน่วยงาน หรือมหาวิทยาลัย หรือมีเอกสารยืนยันเป็นลายลักษณ์อักษร}
\label{\detokenize{3service:id12}}
การคิดภาระงานการบริการวิชาการให้พิจารณาตามสัดส่วน ของการมีส่วนร่วมในงานบริการทางวิชาการนั้นๆ โดยต้องมีเอกสารยืนยันการมีสัดส่วนผลงานจากผู้มีส่วนร่วมทุกคน
\begin{itemize}
\item {} 
สำหรับการบริการวิชาการเชิงพาณิชย์ที่มียอดงบประมาณเกินห้าแสนบาท แต่ไม่เกินหนึ่งล้านบาท ให้คิดภาระงานในอัตรา 1.5 เท่าของภาระงานที่ปรากฏใน 3.2.1 ถึง 3.2.9

\item {} 
หากยอดงบประมาณเกินหนึ่งล้านบาท ให้คิดภาระงานในอัตรา 2 เท่าของภาระงานที่ปรากฏใน 3.2.1 ถึง 3.2.9

\end{itemize}


\subsection{3.2.1  การจัดฝึกอบรม ประชุม และสัมมนา}
\label{\detokenize{3service:id13}}
2 ชั่วโมงต่อสัปดาห์

โดยให้คิดตามกิจกรรมหรือโครงการ (เฉพาะผู้รับผิดชอบโครงการ/กิจกรรม)


\subsection{3.2.2  การค้นคว้า สำรวจ วิเคราะห์ ทดสอบตรวจสอบและตรวจซ่อม}
\label{\detokenize{3service:id14}}
2 ชั่วโมงต่อสัปดาห์

โดยให้คิดตามกิจกรรมหรือโครงการ


\subsection{3.2.3 การวางระบบ ออกแบบ สร้างประดิษฐ์ ผลิตและติดตั้ง}
\label{\detokenize{3service:id15}}
3 ชั่วโมงต่อสัปดาห์

โดยให้คิดตามกิจกรรมหรือโครงการ


\subsection{3.2.4  การให้บริการข้อมูลคำปรึกษาทางวิชาการและวิชาชีพ}
\label{\detokenize{3service:id16}}
3 ชั่วโมงต่อสัปดาห์

การให้บริการข้อมูลคำปรึกษาทางวิชาการและวิชาชีพให้หมายรวมถึง การไปปฏิบัติงานในสถานประกอบการ (Talent Mobility) การรับเป็นที่ปรึกษางานวิจัย การเป็นพี่เลี้ยงงานวิจัยโดยให้คิดตามกิจกรรมหรือโครงการ


\subsection{3.2.5  การให้บริการวิจัยหรือรับจ้างทำวิจัย}
\label{\detokenize{3service:id17}}
1 \sphinxhyphen{} 3.5  ชม.ต่อสัปดาห์

การให้บริการวิจัยหรือรับจ้างวิจัย ให้หมายรวมถึง การรับทำวิจัย และ/หรือ พัฒนาเทคโนโลยี การแก้ปัญหาเชิงเทคนิค การจัดการเทคโนโลยีและนวัตกรรม
โดยให้คิดตามกิจกรรมหรือโครงการ


\subsection{3.2.6  การเขียนทางวิชาการ งานแปลและการผลิตสื่อ}
\label{\detokenize{3service:id18}}
3 ชั่วโมงต่อสัปดาห์

โดยให้คิดตามกิจกรรมหรือโครงการ


\subsection{3.2.7  การให้บริการสารสนเทศและเทคโนโลยีทางการศึกษา}
\label{\detokenize{3service:id19}}
2 ชั่วโมงต่อสัปดาห์

โดยให้คิดตามกิจกรรมหรือโครงการ


\subsection{3.2.8  การเป็นวิทยากร}
\label{\detokenize{3service:id20}}
คิดตามภาระการเป็นวิทยากร


\subsection{3.2.9 การเป็นกรรมการภายนอก}
\label{\detokenize{3service:id21}}
2 ชั่วโมงต่อสัปดาห์

การเป็นกรรมการภายนอก ให้หมายรวมถึง การเป็นกรรมการสมาคมวิชาการหรือวิชาชีพ การเป็นกรรมการสอบวิทยานิพนธ์ การเป็นผู้พิจารณาบทความทางวิชาการ การเป็นผู้พิจาณาผลงานทางวิชาการ โดยให้คิดตามกิจกรรมหรือโครงการ


\section{3.3 งานสอนออกอากาศการศึกษาทางไกล}
\label{\detokenize{3service:id22}}
คิดภาระงานตามชั่วโมงที่ปฏิบัติงานจริงหารด้วย 15 หน่วยสัปดาห์
\begin{description}
\item[{(ก)  วิทยากร}] \leavevmode
ร้อยละ 70

\item[{(ข)  ผู้ช่วยวิทยากร}] \leavevmode
ร้อยละ 30

\end{description}


\chapter{4. ภาระงานบำรุงศิลปวัฒนธรรม}
\label{\detokenize{4culture:id1}}\label{\detokenize{4culture::doc}}\begin{description}
\item[{จำนวนชั่วโมงรวม}] \leavevmode
ไม่เกิน 5 ชั่วโมงต่อสัปดาห์

\end{description}


\bigskip\hrule\bigskip



\section{เงื่อนไขการได้รับการประเมิน}
\label{\detokenize{4culture:id2}}\begin{enumerate}
\sphinxsetlistlabels{\arabic}{enumi}{enumii}{}{.}%
\item {} 
งานที่คณะวิทยาศาสตร์และเทคโนโลยี หรือ ที่มหาวิทยาลัยเทคโนโลยีราชมงคลพระนครดำเนินการจัดขึ้น (ได้แก่ {\hyperref[\detokenize{4culture:id4}]{\sphinxcrossref{\DUrole{std,std-ref}{4.1}}}} และ {\hyperref[\detokenize{4culture:id13}]{\sphinxcrossref{\DUrole{std,std-ref}{4.5}}}})
\begin{itemize}
\item {} 
นำมาคิดภาระงานกี่ครั้งก็ได้ ต่อรอบการประเมิน

\end{itemize}

\item {} 
งานที่คณะอื่นจัด หรืองานจากหน่วยงานภายนอก
\begin{itemize}
\item {} 
นำมาคิดภาระงานได้ไม่เกิน 2 ครั้ง ต่อรอบการประเมิน

\item {} 
ถ้าเข้าร่วมโดยยังไม่ได้รับอนุญาตจากหัวหน้าหน่วยงาน คณาจารย์สามารถนำผลการเข้าร่วมงานดังกล่าวมาคิดภาระงานได้ไม่เกิน 2 ครั้ง ต่อรอบการประเมิน ทั้งนี้ภายหลังการเข้าร่วมงานให้คณาจารย์เร่งจัดทำบันทึกข้อความรายงานหัวหน้าหน่วยงานโดยทันที

\end{itemize}

\item {} 
วันสำคัญของชาติ วันสำคัญทางศาสนา เทศกาลตามประเพณี และเทศกาลตามที่รัฐบาลประกาศ สามารถนำมาคิดภาระงานได้ 1 ครั้ง ต่อวันสำคัญ และ/หรือ เทศกาลนั้นๆ (ถึงแม้ว่างาน และ/หรือ กิจกรรมดังกล่าวข้างต้นจะมีการจัดงาน และ/หรือ กิจกรรมหลายครั้งหรือหลายวันในวันสำคัญ และ/หรือ เทศกาลนั้น)

\end{enumerate}


\section{สิ่งจำเป็นในหลักฐานการประเมิน}
\label{\detokenize{4culture:id3}}\begin{enumerate}
\sphinxsetlistlabels{\arabic}{enumi}{enumii}{}{.}%
\item {} 
\sphinxstylestrong{ต้อง} ระบุวันเวลาอย่างชัดเจน

\item {} 
งานที่คณะวิทยาศาสตร์และเทคโนโลยี หรือ ที่มหาวิทยาลัยเทคโนโลยีราชมงคลพระนครดำเนินการจัดขึ้น (ได้แก่ {\hyperref[\detokenize{4culture:id4}]{\sphinxcrossref{\DUrole{std,std-ref}{4.1}}}} และ {\hyperref[\detokenize{4culture:id13}]{\sphinxcrossref{\DUrole{std,std-ref}{4.5}}}})
\begin{itemize}
\item {} 
ประกาศจากคณะหรือมหาวิทยาลัย

\item {} 
ใบลงทะเบียนเข้าร่วมโครงการที่ระบุเวลาและมีลายเซ็น

\end{itemize}

\item {} 
งานที่คณะอื่นจัด หรืองานจากหน่วยงานภายนอก
\begin{itemize}
\item {} 
หลักฐาน 2 ฉบับ ดังนี้
\begin{itemize}
\item {} 
ก่อนวันงาน: บันทึกข้อความขออนุญาตจากหัวหน้าหน่วยงาน

\item {} 
หลังวันงาน: หลักฐานการเข้าร่วม เช่น ใบลงทะเบียนที่มีลายเซ็น รูปถ่าย

\end{itemize}

\item {} 
ถ้าเข้าร่วมโดยยังไม่ได้รับอนุญาตจากหัวหน้าหน่วยงาน คณาจารย์สามารถนำผลการเข้าร่วมงานดังกล่าวมาคิดภาระงานได้ไม่เกิน 2 ครั้ง ต่อรอบการประเมิน ทั้งนี้ภายหลังการเข้าร่วมงานให้คณาจารย์เร่งจัดทำบันทึกข้อความรายงานหัวหน้าหน่วยงานโดยทันที

\end{itemize}

\end{enumerate}


\bigskip\hrule\bigskip



\section{4.1 การเข้าร่วมในกิจกรรม/โครงการทำนุบำรุงศิลปวัฒนธรรมของมหาวิทยาลัย หรือหน่วยงานภายนอก}
\label{\detokenize{4culture:id4}}\label{\detokenize{4culture:id5}}\begin{description}
\item[{การคำนวณภาระงาน}] \leavevmode
0.5 ชั่วโมงต่อสัปดาห์

\end{description}


\subsection{ประเภทงาน}
\label{\detokenize{4culture:id6}}\begin{description}
\item[{ทำบุญด้วยตนเอง}] \leavevmode
นำมาคิดภาระงานได้ไม่เกิน 6 ครั้ง ต่อรอบการประเมิน โดยไม่นับรวมกับหน่วยงานภายนอก และต้องไม่ตรงกับวันทีเข้าร่วมกิจกรรมอื่นทั้งภายในและภายนอก

\end{description}


\section{4.2 การเป็นผู้รับผิดชอบในกิจกรรม/โครงการทำนุบำรุงศิลปวัฒนธรรมภายในประเทศ}
\label{\detokenize{4culture:id7}}\label{\detokenize{4culture:id8}}\begin{description}
\item[{การคำนวณภาระงาน}] \leavevmode
1 ชั่วโมงต่อสัปดาห์

\end{description}


\section{4.3 การเข้าร่วมในกิจกรรม/โครงการทำนุบำรุงศิลปวัฒนธรรมภายนอกประเทศ}
\label{\detokenize{4culture:id9}}\label{\detokenize{4culture:id10}}\begin{description}
\item[{การคำนวณภาระงาน}] \leavevmode
1 ชั่วโมงต่อสัปดาห์

\end{description}


\section{4.4 การเป็นผู้รับผิดชอบในกิจกรรม/โครงการทำนุบำรุงศิลปวัฒนธรรมภายนอกประเทศ}
\label{\detokenize{4culture:id11}}\label{\detokenize{4culture:id12}}\begin{description}
\item[{การคำนวณภาระงาน}] \leavevmode
1.5 ชั่วโมงต่อสัปดาห์

\end{description}


\section{4.5 การเข้าร่วมในกิจกรรม/โครงการทำนุบำรุงศิลปวัฒนธรรมของคณะวิทยาศาสตร์และเทคโนโลยี}
\label{\detokenize{4culture:id13}}\label{\detokenize{4culture:id14}}\begin{description}
\item[{การคำนวณภาระงาน}] \leavevmode
0.5 ชั่วโมงต่อสัปดาห์

\end{description}


\bigskip\hrule\bigskip



\chapter{5. ภาระงานอื่น ๆ}
\label{\detokenize{5etc:id1}}\label{\detokenize{5etc::doc}}\begin{description}
\item[{จำนวนชั่วโมงรวม}] \leavevmode
ไม่เกิน 5 ชั่วโมงต่อสัปดาห์

\end{description}

งานอื่น ๆ คืองานที่นอกเหนือจาก งานสอน งานวิจัยและวิชาการอื่น งานบริการวิชาการ และงานทำนุบำรุงศิลปวัฒนธรรม


\section{เงื่อนไขการได้รับการประเมิน}
\label{\detokenize{5etc:id2}}

\section{สิ่งจำเป็นในหลักฐานการประเมิน}
\label{\detokenize{5etc:id3}}\begin{enumerate}
\sphinxsetlistlabels{\arabic}{enumi}{enumii}{}{.}%
\item {} \begin{description}
\item[{งานพัฒนาตัวเอง}] \leavevmode\begin{enumerate}
\sphinxsetlistlabels{\arabic}{enumii}{enumiii}{}{.}%
\item {} 
\sphinxstylestrong{ต้อง} ทำบันทึกข้อความล่วงหน้า

\item {} 
\sphinxstylestrong{ต้อง} ระบุระยะเวลาอย่างชัดเจน คณะวิทยาศาสตร์และเทคโนโลยีขอสงวนสิทธิ์ในการดำเนินการ

\end{enumerate}

\end{description}

\end{enumerate}


\section{5.1 งานบริการจัดการสอนมากกว่า 1 ศูนย์การศึกษา และ/หรือ นอกศูนย์ที่ตั้งของคณะ}
\label{\detokenize{5etc:id4}}\begin{description}
\item[{(ก) การสอนนอกศูนย์ที่ตั้งของคณะ}] \leavevmode
1 ชั่วโมงต่อสัปดาห์ (ไม่รวมการคุมสอบต่างศูนย์การศึกษา)

\item[{(ข) 2 ศูนย์}] \leavevmode
2 ชั่วโมงต่อสัปดาห์

\item[{(ค) 3 ศูนย์}] \leavevmode
3 ชั่วโมงต่อสัปดาห์

\item[{(ง) 4 ศูนย์}] \leavevmode
4 ชั่วโมงต่อสัปดาห์

\item[{(จ) การคุมสอบต่างศูนย์}] \leavevmode
0.5 ชั่วโมงต่อสัปดาห์

\end{description}


\section{5.2 งานพัฒนานักศึกษา}
\label{\detokenize{5etc:id5}}

\subsection{5.2.1 งานอาจารย์ที่ปรึกษา}
\label{\detokenize{5etc:id6}}
งานอาจารย์ที่ปรึกษา หมายถึง อาจารย์ที่ปรึกษาชั้นปีและอาจารย์ที่ปรึกษาโครงการ/กิจกรรมของนักศึกษา เช่น กิจกรรมในงานสโมสรนักศึกษา กิจกรรมค่ายอาสา กิจกรรมกีฬามหาวิทยาลัย เป็นต้น

2 ชั่วโมงต่อสัปดาห์

ให้คำนวณตามจำนวนงาน/กิจกรรม


\subsection{5.2.2 งาน/กิจกรรมพัฒนานักศึกษานอกพื้นที่}
\label{\detokenize{5etc:id7}}
1 ชั่วโมงต่อสัปดาห์

ให้คำนวณตามจำนวนงานหรือกิจกรรม


\subsection{5.2.3 งาน/กิจกรรมพัฒนานักศึกษาในพื้นที่}
\label{\detokenize{5etc:id8}}
0.5 ชั่วโมงต่อสัปดาห์

ให้คำนวณตามจำนวนงานหรือกิจกรรม


\section{5.3 งานพัฒนาองค์กร}
\label{\detokenize{5etc:id9}}
งานพัฒนาองค์กร หมายถึง งาน/ผลงานที่ก่อให้เกิดการพัฒนาและสร้างคุณประโยชน์ให้แก่องค์กร (เช่น การได้รับกำหนดตำแหน่งทางวิชาการ การเพิ่มคุณวุฒิทางวิชาการ/วิชาชีพ การสร้างชื่อเสียงและส่งเสริมภาพลักษณ์ขององค์กร)
\begin{description}
\item[{(ก) สร้างชื่อเสียงหรือได้รับการยอมรับระดับนานาชาติ}] \leavevmode
3 ชั่วโมงต่อสัปดาห์

\item[{(ข) สร้างชื่อเสียงหรือได้รับการยอมรับระดับชาติ}] \leavevmode
2 ชั่วโมงต่อสัปดาห์

\item[{(ค) สร้างคุณประโยชน์ให้กับคณะ}] \leavevmode
1 ชั่วโมงต่อสัปดาห์

\end{description}

งานสร้างคุณประโยชน์ให้กับคณะ หมายถึง
\begin{enumerate}
\sphinxsetlistlabels{\arabic}{enumi}{enumii}{}{.}%
\item {} 
กรรมการเกี่ยวกับงานพัสดุ บัญชี และการเงิน

\item {} 
กรรมการที่มีความเสี่ยง

\end{enumerate}
\begin{description}
\item[{(ง) งานพัฒนานวัตกรรมการจัดการองค์กร}] \leavevmode
3 ชั่วโมงต่อสัปดาห์

\item[{(จ) ภาระงานตาม KPI ของคณะ}] \leavevmode
1 ชั่วโมงต่อสัปดาห์

\end{description}

\begin{sphinxadmonition}{important}{Important:}
ผู้ดำรงตำแหน่งทางวิชาการ และผู้มีคุณวุฒิ ปริญญาเอก นับเป็นภาระงานตาม KPI สามารถนับภาระงานได้ \sphinxstylestrong{ทุกรอบ} การประเมิน
\end{sphinxadmonition}
\begin{description}
\item[{(ฉ) กรรมการ/โครงการ/กิจกรรมที่ได้รับความเห็นชอบจากคณะกรรมการกำหนดภาระงานขั้นต่ำ ของคณาจารย์คณะวิทยาศาสตร์และเทคโนโลยี}] \leavevmode
1 ชั่วโมงต่อสัปดาห์

\end{description}

ให้คิดตามจำนวนโครงการ/กิจกรรม


\section{5.4 งานพัฒนาตนเอง}
\label{\detokenize{5etc:id10}}
งานพัฒนาตนเอง หมายถึง การไปฝึกอบรมให้เป็นไปตามประกาศกระทรวงศึกษาธิการ เรื่อง มาตรฐานการอุดมศึกษา พ.ศ. 2561 โดยได้รับความเห็นชอบจากหัวหน้าหน่วยงาน ทั้งนี้ ให้หมายรวมถึง การลงทะเบียนเรียนหรือฝึกอบรมในหลักสูตรออนไลน์ที่ได้รับใบรับรองให้คิดตามจำนวนโครงการ/กิจกรรม

คิดคำนวณภาระงานตาม ชม. ที่ปฏิบัติงานจริงหารด้วย 15


\section{5.5 งานจิตอาสา}
\label{\detokenize{5etc:id11}}
งานจิตอาสา หมายถึง กิจกรรมการบำเพ็ญประโยชน์ในคณะ ในมหาวิทยาลัย ในหน่วยงานภาครัฐ เอกชน และชุมชน
ให้คิดตามกิจกรรม/โครงการ
\begin{description}
\item[{ก) บำเพ็ญประโยชน์ในคณะ}] \leavevmode
0.5 ชั่วโมงต่อสัปดาห์

\item[{(ข) บำเพ็ญประโยชน์ในมหาวิทยาลัย}] \leavevmode
1 ชั่วโมงต่อสัปดาห์

\item[{(ค) บำเพ็ญประโยชน์ในหน่วยงานภาครัฐ และเอกชน}] \leavevmode
1.5 ชั่วโมงต่อสัปดาห์

\end{description}

บำเพ็ญประโยชน์ในหน่วยงานภาครัฐ และเอกชน ให้หมายรวมถึง การบำเพ็ญประโยชน์ ร่วมกับ หน่วยงานภาครัฐ เอกชน และชุมชน โดยภาครัฐ หมายถึง หน่วยงานรัฐที่มิใช่หน่วยงานภายในของมหาวิทยาลัยเทคโนโลยีราชมงคลพระนคร ทั้งนี้ ภาระงานจิตอาสาที่จะนำมาพิจารณาได้จะต้องได้รับอนุญาตจากคณบดี


\chapter{6. ภาระงานของผู้ดำรงตำแหน่งวิชาการ}
\label{\detokenize{6academicposition:id1}}\label{\detokenize{6academicposition::doc}}
ผลงานของผู้ดำรงตำแหน่งทางวิชาการ ตาม ประกาศมหาวิทยาลัยเทคโนโลยี
ราชมงคลพระนคร เรื่อง เกณฑ์ภาระงานขั้นต่ำฯ พ.ศ. 2559 (ข้อ 5)

ใช้หลักฐานที่เป็นรูปแบบไฟล์อิเล็กทรอนิกส์ (PDF หรือ ภาพถ่าย) ตามหัวข้อผลงานในประกาศมหาวิทยาลัยเทคโนโลยีราชมงคลพระนคร เรื่อง เกณฑ์ภาระงานขั้นต่ำของคณาจารย์ประตำ พ.ศ. 2559 ข้อ 5 1. ผลงานของผู้ดำรงตำแหน่งทางวิชาการ


\section{เงื่อนไขการได้รับการประเมิน}
\label{\detokenize{6academicposition:id2}}\begin{itemize}
\item {} 
ในส่วนของบทความจากผลงานวิจัย และ/หรือ บทความทางวิชาการ ซึ่งการทำข้อตกลงคณาจารย์ต้องระบุ 100\% เท่านั้น ไม่สามารถส่ง Draft Manuscript

\item {} 
ส่วนผลงานอื่นสามารถระบุ \% ในการจัดทำผลงานทำต่อปี

\item {} 
ผลงานของผู้ดำรงตำแหน่งทางวิชาการที่ระบุไม่ถึง 100\% ไม่ต้องดำเนินการขอเผยแพร่ แต่ให้จัดทำบันทึกข้อความเสนอคณบดีรับทราบ

\end{itemize}


\section{สิ่งจำเป็นในหลักฐานการประเมิน}
\label{\detokenize{6academicposition:id3}}\begin{enumerate}
\sphinxsetlistlabels{\arabic}{enumi}{enumii}{}{.}%
\item {} 
\sphinxstylestrong{ต้อง} ระบุวันเวลาอย่างชัดเจน คณะวิทยาศาสตร์และเทคโนโลยีขอสงวนสิทธิ์ในการดำเนินการ

\end{enumerate}


\section{6.1 งานของผู้ดำรงตำแหน่งผู้ช่วยศาสตราจารย์}
\label{\detokenize{6academicposition:id4}}
ุมีบทความจากผลงานวิจัย ที่ได้รับการตีพิมพ์เผยแพร่ในวารสารวิชาการที่มีกระบวนการตรวจสอบผลงานทางวิชาการโดยคณะกรรมการ (Peer Review) ก่อนตีพิมพ์ และเป็นวารสารที่ยอมรับในวงการวิชาการสาขานั้น ๆ หรือได้นำเสนอในการประชุมวิชาการ พร้อมทั้งเสนอผลงานฉบับสมบูรณ์ หรือ ผลงานในลักษณะอื่น ที่เทียบเท่า ปีละ 1 เรื่อง หรือ บทความวิชาการในลักษณะอื่น เช่น บทปริทรรศน์ เฉลี่ยปีละ 2 เรื่อง หรือมีผลงานในลักษณะอื่นที่เทียบเท่าเป็นอย่างหนึ่งอย่างใดต่อไปนี้
\begin{enumerate}
\sphinxsetlistlabels{\arabic}{enumi}{enumii}{}{.}%
\item {} 
นวัตกรรมหรือสิ่งประดิษฐ์ 1 ผลงาน ต่อปี

\item {} 
งานบริการวิชาการตามที่มหาวิทยาลัยหรือคณะเห็นชอบไม่น้อยกว่า 3 ครั้ง ต่อปี

\item {} 
เอกสารประกอบการสอน เอกสารคำสอน หนังสือ ตำรา ที่เกี่ยวกับสาขาวิชา 1 ผลงาน ต่อปี

\item {} 
คู่มือปฏิบัติการ 1 รายวิชา

\item {} 
สิทธิบัตร อนุสิทธิบัตร ลิขสิทธิ์ 1 คำขอ ต่อปี

\end{enumerate}


\section{6.2 งานของผู้ดำรงตำแหน่งรองศาสตราจารย์}
\label{\detokenize{6academicposition:id5}}
มีบทความจากผลงานวิจัยหรือผลงานในลักษณะอื่นที่เทียบเท่า ที่ได้รับการตีพิมพ์เผยแพร่ในวารสารวิชาการที่มีกระบวนการตรวจสอบผลงานทางวิชาการโดยคณะกรรมการ (Peer Review) ก่อนตีพิมพ์ และเป็นวารสารที่ยอมรับในวงการวิชาการสาขานั้น ๆ หรือ ผลงานในลักษณะอื่น ที่เทียบเท่า ปีละ 2 เรื่อง หรือ มีผลงานในลักษณะอื่นที่เทียบเท่าเป็นอย่างหนึ่งอย่างใดต่อไปนี้
\begin{enumerate}
\sphinxsetlistlabels{\arabic}{enumi}{enumii}{}{.}%
\item {} 
นวัตกรรมหรือสิ่งประดิษฐ์ โดยเฉลี่ย 3 ผลงาน ต่อ 2 ปี

\item {} 
งานบริการวิชาการตามที่มหาวิทยาลัยหรือคณะเห็นชอบไม่น้อยกว่า 6 ครั้ง ต่อปี

\item {} 
หนังสือ ตำรา ที่เกี่ยวกับสาขาวิชา 1 ผลงาน ต่อปี

\item {} 
สิทธิบัตร อนุสิทธิบัตร ลิขสิทธิ์ 1 คำขอ ต่อปี

\end{enumerate}


\section{6.3 งานของผู้ดำรงตำแหน่งศาสตราจารย์}
\label{\detokenize{6academicposition:id6}}
มีบทความจากผลงานวิจัยหรือผลงานในลักษณะอื่นที่เทียบเท่า ที่ได้รับการตีพิมพ์เผยแพร่ในวารสารวิชาการที่มีกระบวนการตรวจสอบผลงานทางวิชาการโดยคณะกรรมการ (Peer Review) ก่อนตีพิมพ์ และเป็นวารสารที่ยอมรับในวงการวิชาการสาขานั้น ๆ หรือ ผลงานในลักษณะอื่น ที่เทียบเท่า ปีละ 3 เรื่อง โดยต้องเป็นวารสารในระดับนานาชาติ อย่างน้อยปีละ 1 เรื่อง หรือ มีผลงานในลักษณะอื่นที่เทียบเท่าเป็นอย่างหนึ่งอย่างใดต่อไปนี้
\begin{enumerate}
\sphinxsetlistlabels{\arabic}{enumi}{enumii}{}{.}%
\item {} 
นวัตกรรมหรือสิ่งประดิษฐ์ 2 ผลงาน ต่อปี

\item {} 
งานบริการวิชาการตามที่มหาวิทยาลัยหรือคณะเห็นชอบไม่น้อยกว่า 9 ครั้ง ต่อปี

\item {} 
เอกสารประกอบการสอน เอกสารคำสอน หนังสือ ตำรา ที่เกี่ยวกับสาขาวิชา 1 ผลงาน ต่อปี

\item {} 
สิทธิบัตร อนุสิทธิบัตร ลิขสิทธิ์ 1 คำขอ ต่อไป

\end{enumerate}


\chapter{7. ภาระงานของกลุ่มผู้บริหารและการเป็นคณะกรรมการ}
\label{\detokenize{7executive:id1}}\label{\detokenize{7executive::doc}}
ภาระงานของกลุ่มผู้บริหารและการเป็นคณะกรรมการ ตามประกาศฯ มหาวิทยาลัย เรื่องเกณฑ์ภาระงานขั้นต่ำของคณาจารย์ประจำ พ.ศ.2559 ข้อ 4 (4) \sphinxhyphen{} (12) ดังนี้
\begin{enumerate}
\sphinxsetlistlabels{\arabic}{enumi}{enumii}{}{.}%
\item {} 
อธิการบดี คิดเป็นภาระงาน 35 ชั่วโมงต่อสัปดาห์

\item {} 
รองอธิการบดี คิดเป็นภาระงาน 30 ชั่วโมงต่อสัปดาห์

\item {} 
ผู้อำนายการสำนักงานอธิการบดี คณะบีด ผู้อำนวยการสถาบัน ผู้อำนวยการสำนัก ผู้อำนวยการวิทยาลัย ผู้อำนวยการกอง หรือ หัวหน้าหน่วยงานที่เรียกชื่ออย่างอื่นที่มีฐานะเทียบเท่าคณะ หรือกอง ที่มีภาระงานบริหารเต็มเวลา คิดเป็นภาระงาน 30 ชั่วโมงต่อสัปดาห์

\item {} 
ผู้ช่วยอธิการบดี คิดเป็นภาระงาน 20 ชั่วโมงต่อสัปดาห์

\item {} 
รองคณบดี รองผู้อำนวยการวิทยาลัย รองผู้อำนวยการสถาบัน รองผู้อำนวยการสำนัก หรือรองหัวหน้าหน่วยงานที่เรียกชื่ออย่างอื่นที่มีฐานะเทียบเท่าคณะ คิดเป็นภาระงาน 20 ชั่วโมงต่อสัปดาห์

\item {} 
หัวหน้าสาขาวิชา คิดเป็นภาระงาน 12 ชั่วโมงต่อสัปดาห์

\item {} 
ผู้ช่วยคณบดี ผู้ช่วยผู้อำนวยการวิทยาลัย ผู้ช่วยผู้อำนวยการสถาบัน ผู้ช่วยผู้อำนวยการสำนัก หรือผู้ช่วยหัวหน้าหน่วยงานที่เรียกชื่ออย่างอื่น ที่มีฐานะเทียบเท่าคณะ คิดเป็นภาระงาน 12 ชั่วโมงต่อสัปดาห์

\item {} 
ประสภานสภาคณาจารย์ และข้าราชการ คิดเป็นภาระงาน 6 ชั่วโมงต่อสัปดาห์

\item {} 
กรรมการในสภามหาวิทยาลัย สภาคณาจารย์และข้าราชการ สภาวิชาการ คิดเป็นภาระงาน 3 ชั่วโมงต่อสัปดาห์

\item {} 
เลขาณุการในสภามหาวิทยาลัย สภาคณาจารย์และข้าราชการ สภาวิชาการ คิดเป็นภาระงาน 5 ชั่วโมงต่อสัปดาห์

\item {} 
อาจารย์ผู้รับผิดชอบหลักสูตร คิดเป็นภาระงาน 8 ชั่วโมงต่อสัปดาห์

\item {} 
หัวหน้ากลุ่มวิชา หัวหน้างาน คิดเป็นภาระงาน 6 ชั่วโมงต่อสัปดาห์

\end{enumerate}

สำหรับผู้บริหารตามข้อ 2 และ 3 ภาระงานอีก 5 ภาระงาน ให้เลือกทำภาระงานสอน หรือภาระงานวิจัยและวิชาการอื่น นอกจากนี้คณาจารย์ที่เป็นคณะกรรมการโดยตำแหน่งจะไม่นำมาคิดเป็นภาระงาน


\chapter{อภิธานศัพท์}
\label{\detokenize{glossary:id1}}\label{\detokenize{glossary::doc}}\begin{description}
\item[{เอกสารประกอบการสอน\index{เอกสารประกอบการสอน@\spxentry{เอกสารประกอบการสอน}|spxpagem}\phantomsection\label{\detokenize{glossary:term-0}}}] \leavevmode
เอกสารที่ใช้ประกอบการสอน

\item[{เอกสารคำสอน\index{เอกสารคำสอน@\spxentry{เอกสารคำสอน}|spxpagem}\phantomsection\label{\detokenize{glossary:term-1}}}] \leavevmode
เอกสารคำสอน

\item[{หนังสือ\index{หนังสือ@\spxentry{หนังสือ}|spxpagem}\phantomsection\label{\detokenize{glossary:term-2}}}] \leavevmode
หนังสือที่ต้องตีพิมพ์

\item[{ตำรา\index{ตำรา@\spxentry{ตำรา}|spxpagem}\phantomsection\label{\detokenize{glossary:term-3}}}] \leavevmode
ตีพิมพ์

\item[{สื่อการสอน\index{สื่อการสอน@\spxentry{สื่อการสอน}|spxpagem}\phantomsection\label{\detokenize{glossary:term-4}}}] \leavevmode
สื่อการสอนหมายถึง สื่อที่อาจารย์ผู้สอนพัฒนาด้วยตนเอง ได้แก่
\begin{itemize}
\item {} 
{\hyperref[\detokenize{glossary:term-0}]{\sphinxtermref{\DUrole{xref,std,std-term}{เอกสารประกอบการสอน}}}}

\item {} 
{\hyperref[\detokenize{glossary:term-1}]{\sphinxtermref{\DUrole{xref,std,std-term}{เอกสารคำสอน}}}}

\item {} 
{\hyperref[\detokenize{glossary:term-2}]{\sphinxtermref{\DUrole{xref,std,std-term}{หนังสือ}}}}

\item {} 
{\hyperref[\detokenize{glossary:term-3}]{\sphinxtermref{\DUrole{xref,std,std-term}{ตำรา}}}}

\item {} 
เอกสารนำเสนอฉบับอิเล็กทรอนิกส์ (เช่น PowerPoint, Google Slides)

\item {} 
บทเรียนในระบบอินเตอร์เน็ต (เช่น MOOC, e\sphinxhyphen{}Learning)

\end{itemize}

สื่อการสอน \sphinxstylestrong{ไม่รวม} ถึงสื่อต่อไปนี้
\begin{itemize}
\item {} 
ระบบจัดการเรียนการสอน เช่น ระบบ LMS, Google Classroom

\item {} 
ซอฟต์แวร์ video call เช่น Google meet, Microsoft Team, Zoom เป็นต้น

\item {} 
ซอฟต์แวร์ที่นำมาใช้ในประกอบการสอน เช่น MATLAB, Desmos

\item {} 
สื่อประกอบการสอนอื่นๆ เช่น อุปกรณ์ สารเคมี VDO หนังสือที่ไม่ได้พัฒนาด้วยตนเอง เป็นต้น

\end{itemize}

\item[{ผู้มีส่วนร่วมในโครงการ\index{ผู้มีส่วนร่วมในโครงการ@\spxentry{ผู้มีส่วนร่วมในโครงการ}|spxpagem}\phantomsection\label{\detokenize{glossary:term-5}}}] \leavevmode
หัวหน้าโครงการ วิทยากร ผู้ช่วยวิทยากร หรือ ผู้รับผิดชอบโครงการ

\item[{มีส่วนร่วมกับหน่วยงานภายนอก\index{มีส่วนร่วมกับหน่วยงานภายนอก@\spxentry{มีส่วนร่วมกับหน่วยงานภายนอก}|spxpagem}\phantomsection\label{\detokenize{glossary:term-6}}}] \leavevmode
หาประกาศของมหาลัย

\item[{แบบรับรองการเผยแพร่ผลงานทางวิชาการ\index{แบบรับรองการเผยแพร่ผลงานทางวิชาการ@\spxentry{แบบรับรองการเผยแพร่ผลงานทางวิชาการ}|spxpagem}\phantomsection\label{\detokenize{glossary:term-7}}}] \leavevmode
แบบรับรองที่คณบดีลงนาม

\item[{แบบรับรองภาระงานทางวิชาการ\index{แบบรับรองภาระงานทางวิชาการ@\spxentry{แบบรับรองภาระงานทางวิชาการ}|spxpagem}\phantomsection\label{\detokenize{glossary:term-8}}}] \leavevmode
แบบรับรองการเผยแพร่ภาระงานทางวิชาการ ของผู้ดำรงตำแหน่ง ผู้ช่วยศาสตราจารย์ รองศาสตราจารย์ และศาสตราจารย์

\item[{หน่วยงานภายใน\index{หน่วยงานภายใน@\spxentry{หน่วยงานภายใน}|spxpagem}\phantomsection\label{\detokenize{glossary:term-9}}}] \leavevmode
หน่วยงานในมหาวิทยาลัย

\item[{หน่วยงานภายนอก\index{หน่วยงานภายนอก@\spxentry{หน่วยงานภายนอก}|spxpagem}\phantomsection\label{\detokenize{glossary:term-10}}}] \leavevmode
หน่วยงานนอกมหาวิทยาลัย

\item[{การบูรณาการ\index{การบูรณาการ@\spxentry{การบูรณาการ}|spxpagem}\phantomsection\label{\detokenize{glossary:term-11}}}] \leavevmode
การผสมกลมกลืนของแผน กระบวนการ สารสนเทศ กรจัดสรรทรัพยากร การปฏิบัติการ ผลลัพธ์ และการวิเคราะห์ เพื่อสนับสนุนเป้าประสงค์ที่สำคัญของสถานบัน (organization\sphinxhyphen{}wide goal) การบูรณาการที่มีประสิทธิผลเป็นมากกว่าความสอดคล้องไปในแนวทางเดียวกัน (alignment) ซึ่งการดำเนินการของแต่ละองค์ประกอบภายในระบบการจัดการ ผลการดำเนินการ มีความเชื่อมโยงกันเป็นหนึ่งเดียวอย่างสมบูรณ์

(จาก คู่มือการประกันคุณภาพการศึกษาภายใน ระดับอุดมศึกษา พ.ศ. )2557

\end{description}


\chapter{คำถามที่พบบ่อย}
\label{\detokenize{faq:id1}}\label{\detokenize{faq::doc}}
\begin{sphinxadmonition}{note}{Note:}
Q:

A:
\end{sphinxadmonition}


\chapter{เอกสารประกอบ}
\label{\detokenize{official_documents:id1}}\label{\detokenize{official_documents::doc}}
ประกาศคณะวิทยาศาสตร์และเทคโนโลยีมหาวิทยาลัยเทคโนโลยีราชมงคลพระนคร เรื่อง เกณฑ์ภาระงานขั้นต่ำของคณาจารย์ ประจำปี 2559
\sphinxcode{\sphinxupquote{pdf}}

หลักเกณฑ์และวิธีการคำนวณภาระงานขั้นต่ำของคณาจารย์ประจำ พ.ศ. 2563
\sphinxcode{\sphinxupquote{pdf}}

ประกาศ ก.พ.อ. เรื่อง หลักเกณฑ์และวิธีการพิจารณาแต่งตั้งบุคคลให้ดำรงตำแหน่ง ผู้ช่วยศาสตราจารย์ รองศาสตราจารย์ และศาสตราจารย์ พ.ศ. 2560
\sphinxcode{\sphinxupquote{pdf}}

ประกาศ ก.พ.อ. เรื่อง หลักเกณฑ์และวิธีการพิจารณาแต่งตั้งบุคคลให้ดำรงตำแหน่ง ผู้ช่วยศาสตราจารย์ รองศาสตราจารย์ และศาสตราจารย์ พ.ศ. 2563
\sphinxcode{\sphinxupquote{pdf}}



\renewcommand{\indexname}{Index}
\printindex
\end{document}